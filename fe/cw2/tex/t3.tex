The aspect ratios for Jianer are
\begin{itemize}
\item Quad Element
  \begin{description}
  \item[{AR}] 2.67
  \item[{x}] 0.75 in
  \item[{y}] 2.00 in
  \end{description}
\item Tri Element
  \begin{description}
  \item[{AR}] 6
  \item[{x}] 48 in
  \item[{y}] 8 in
  \end{description}
\end{itemize}
\def\g#1{\MyGet{t3.#1}}
To calculate the vertical displacement at tip, we start with
\def\i{\text{in}}
\begin{align*}
  L &= \g{L} \i\\
  b &= \g{b} \i\\
  h &= \g{h} \i\\
  P &= \g{P} \text{kip}.
\end{align*}
Therefore for the analytical expression, we have
\begin{align*}
  A &= bh = \g{b} \i \times \g{h} \i = \g{A} \i^2\\
  I &= \frac{bh^3}{12} = \frac{\g{b} \times \g{h}^3}{12} = \g{I} \text{m}^4 \\
  E &= \g{E} \text{ksi}\\
  G &= \g{G} \text{ksi}\\
  k &= \g{k}
\end{align*}
Therefore,
\begin{align*}
  V_y &= \frac{PL^3}{3EI} + \frac{PL}{kAG} \\
      &= \frac{\g{P} \times \g{L}^3}{3 \times \g{E} \times \g{I}} +
        \frac{\g{P} \times \g{L}}{\g{k} \times \g{A} \times \g{G}} \\
      &= \g{Vy} \i
\end{align*}
The comparison of results are shown below:
\begin{center}
  \begin{tabular}{ll}
    Source & Displacement \(v_y\) (in)\\
    \hline
    Hand Calc   & \g{Vy} \\
    GSA 64-by-4 quad elements & 1.213       \\
    GSA 1-by-1 tri elements   & 0.0538
  \end{tabular}
\end{center}
We observe that, the 64-by-4 quad element mesh gives a much better closer result
than the 1-by-1 tri element mesh. (The later is not acceptable in fact.) So, a
good mesh is a fine mesh. I personally think a mesh is fine enough if the result
would not be much different by further refining the mesh.

For example, a 128-by-16 quad element mesh would be a fine enough mesh for our
example. And our 1-by-1 tri element mesh is an example of a coarse mesh.
