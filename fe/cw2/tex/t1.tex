\def\g#1{\MyGet{t1.#1}}
The given parameters are:
\def\m{\text{m}}
\begin{align*}
  L &= \g{L} \m\\
  b &= \g{b} \m\\
  h &= \g{h} \m\\
  P &= \g{P} \text{kN}
\end{align*}
Therefore for the analytical expression, we have
\begin{align*}
  A &= bh = \g{b} \m \times \g{h} \m = \g{A} \m^2\\
  I &= \frac{bh^3}{12} = \frac{\g{b} \times \g{h}^3}{12} = \g{I} \text{m}^4 \\
  E &= \g{E} \text{kNm}^{-2}\\
  G &= \g{G} \text{kNm}^{-2}\\
  k &= \g{k}
\end{align*}
Therefore,
\begin{align*}
  V_y &= \frac{PL^3}{3EI} + \frac{PL}{kAG} \\
      &= \frac{\g{P} \times \g{L}^3}{3 \times \g{E} \times \g{I}} +
        \frac{\g{P} \times \g{L}}{\g{k} \times \g{A} \times \g{G}} \\
      &= \g{Vy} \m = \g{Vy.mm} \text{mm}
\end{align*}
The comparison of results are shown below:
\begin{center}
  \begin{tabular}{ll}
    Source & Displacement \(v_y\) (mm)\\
    \hline
    Hand Calc   & \g{Vy.mm} \\
    GSA k = 5/6 & 0.6451       \\
  GSA k = 0   &   0.5928     \\
  \end{tabular}
\end{center}
It is expected that ignoring the shear deformation (the third case) will give a lower
deformation.

It is also expected that the GSA result
be close to the hand calculation result when the shear deformation is accounted
for.

The results from above table agree with these.

