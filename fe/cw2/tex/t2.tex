\def\g#1{\MyGet{t2.#1}}
The data for Jianer is
\def\m{\text{m}}
\begin{align*}
L = \g{L} \m , b = \g{b} \m , h = \g{h} \m
\end{align*}
therefore we have
\begin{align*}
I = \frac{bh^3}{12} = \frac{\g{b} \times \g{h}^3}{12} = \g{I} \text{m}^4
\end{align*}
from GSA we have
\begin{align*}
  E &= \g{E} \text{kNm}^{-2} \\
  m &= \g{m} \text{kgm}^{-1}
\end{align*}
So,
\begin{align*}
  f_n &= \frac{3.52}{2\pi}\sqrt{\frac{EI}{mL^4}} \\
  &= \g{f1} \times \sqrt{\g{f2}} = \g{fn}
\end{align*}

The comparison with the result from GSA is presented below
\begin{center}\begin{tabular}{|c|c|c|c|}
\hline
Source & Number of element & Frequency of vibration \\
\hline
GSA: mass lumped at nodes &     1 & 4.671 \\
      &     2 & 4.651 \\
      &     3 & 4.649 \\
      &     5 & 4.648 \\
      &    10 & 4.648 \\
\hline
GSA: distributed mass &     1 & 3.244 \\
      &     2 & 4.176 \\
      &     3 & 4.425 \\
      &     5 & 4.565 \\
      &    10 & 4.627 \\
\hline
Hand calculation & & 4.68\\\hline
\end{tabular}\end{center}
As shown, as the number of element increases, the values given by assuming
mass lumped at nodes are getting closer and closer to the hand calculation value.

Also, the values for distributed mass are all close to the hand calculation
value, even there seems to be a small divergence as the number of elements
increases.

