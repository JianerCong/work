\newcommand{\dXX}[2][x]{
  \ensuremath{\frac{\partial^2 #2}{\partial #1^2}}
}
\newcommand{\dX}[2][x]{
  \ensuremath{\frac{\partial #2}{\partial #1}}
}
\newcommand{\dxx}[1]{\dXX{#1}}
\newcommand{\dyy}[1]{\dXX[y]{#1}}
\newcommand{\dx}[1]{\dX{#1}}
\newcommand{\dy}[1]{\dX[y]{#1}}
\newcommand{\phin}{\ensuremath{
    \dX[\mathbf{n}]{\phi} 
  }}
\begin{questionbox}
  The function $\phi(x,y)$ is governed by
\begin{align}
\dxx{\phi} + \dyy{\phi} = 0 \label{eq:poi}
\end{align}
The mesh is shown in the following figure:

\newtcbox{bLabBox}[1][red]{
  on line,arc=0pt,colback=#1!10!white,
  colframe=#1!50!black,
  boxsep=0pt,left=1pt,right=1pt,top=2pt,
  bottom=2pt,boxrule=0pt,bottomrule=1pt,toprule=1pt
}
\newtcbox{nLabBox}[1][red]{
  on line,top=2pt,bottom=2pt,
  left=3pt,right=3pt,
  colframe=#1,
  colback=#1!10!white
}
\begin{tikzpicture}[x=1cm,y=1cm]
  \newcommand{\bLab}[4][(0,0)]{
    % \node[fill=#2,text=white,#3] at #1 {#4};
    \node[#3] at #1 {
      \bLabBox[#2]{#4}
    };
  }

  \begin{scope}[very thick]
    \draw[style=help lines,step=1cm] (0,0) grid (10,6);
    % \tikzstyle{every node}=[draw,circle]

    \draw[draw=\mycola] (0,0) -- (0,6);
    \bLab[(0-0.1,3)]{\mycola}{left}{1}

    \draw[draw=\mycolb] (0,6) -- (10,6);
    \bLab[(4.5,6.1)]{\mycolb}{above}{2}

    \draw[draw=\mycolc] (10,6) -- (10,4);
    \bLab[(10.1,5)]{\mycolc}{right}{3}

    \draw[draw=\mycola] (10,4) -- (7,2) -- (5,0);
    \bLab[(7-0.2,1.2)]{\mycola}{right}{4}

    \draw[draw=\mycolb] (5,0) -- (0,0);
    \bLab[(2.5,-0.1)]{\mycolb}{below}{5}
  \end{scope}

  \begin{scope}
    % draw the mesh
    \draw (0,2) -- +(7,0)
    (5,0) -- ++(0,6)
     -- +(2,-4)
     -- +(5,-2)
     -- +(0,0);

     \foreach \x/\y/\l in {
       0/0/1,5/0/2,0/2/3,5/2/4,
       7/2/5,10/4/6,0/6/7,5/6/8,
       10/6/9}{
       \fill[black] (\x,\y) circle (3pt);
       \node[above right] at (\x,\y) {\l};
     }

     \newcommand{\nLab}[3][(0,0)]{
       % \node[fill=#2,text=white,rectangle] at #1 {#3};
       \node at #1 {\nLabBox[#2]{#3}};
     }

     \foreach \x/\y/\l/\c in {
       2.5/1/a/\mycola,2.5/4/b/\mycolb,
       6/1.5/c/\mycolc,6/3/d/\mycola,
       7/4/e/\mycolb,8/5.5/f/\mycolc
     }{\nLab[(\x,\y)]{\c}{\l}}
  \end{scope}
\end{tikzpicture}

\MySet{bd.col.1}{\mycola}
\MySet{bd.col.2}{\mycolb}
\MySet{bd.col.3}{\mycolc}
\MySet{bd.col.4}{\mycola}
\MySet{bd.col.5}{\mycolb}
\foreach \n/\i in {
  1/\mycola,
  2/\mycolb,
  3/\mycolc,
  4/\mycola,
  5/\mycolb}{
  \MySet{bd.col.\n}{\i}
}

\newcommand{\bd}[1]{\bLabBox[\MyGet{bd.col.#1}]{#1}}
\newcommand{\el}[1]{\nLabBox[\MyGet{el.col.#1}]{#1}}

% \foreach \n/\i in {
%   a/\mycola,
%   b/\mycolb,
%   c/\mycolc,
%   d/\mycola,
%   e/\mycolb,
%   f/\mycolc}{
%   \MySet{el.col.\n}{\i}
% }
% Don't know why it dosn't work   
\MySet{el.col.a}{\mycola}
\MySet{el.col.b}{\mycolb}
\MySet{el.col.c}{\mycolc}
\MySet{el.col.d}{\mycola}
\MySet{el.col.e}{\mycolb}
\MySet{el.col.f}{\mycolc}

Where \bd{1} to \bd{5} are boundaries,
and \el{a} to \el{f} are elements.

And the BCs are:
\def\f#1{\mbox{Along \bd{#1}: }}
\begin{align}
    \f{1} & \phin = 0 &  \f{2} & \phi = au_0 \notag\\
    \f{3} & \phin = 0 & \f{4} & \phi = 0 \notag\\ 
    \f{5} & \phi = 0 && \notag
\end{align}
\end{questionbox}

\subsection*{Answer}\label{sec:q2}
\subsubsection*{Weak formulation}
Let $v$ be an arbitrary test function, the weak formulation of \cref{eq:poi} is
then
\newcommand{\inta}[1]{
  \ensuremath{\int_{\text{Domain}} #1 \text{dA}}
}
\newcommand{\intb}[1]{
  \ensuremath{\int_{\text{Boundary}} #1 \text{ds}}
}

\def\x{
  \ensuremath{\inta{\nabla \phi \bullet \nabla v}}
}
\def\y{
  \ensuremath{\intb{\phin v}}
}


\begin{align}
  0 &= \inta{(\dxx{\phi} + \dyy{\phi}) v} \notag\\
  \intertext{By Gauss-Green}
    &= \inta{-\dx{\phi}\dx{v} - \dy{\phi}\dy{v}} +
      \intb{
      (\dx{\phi}\dx{n} + \dy{\phi}\dy{n})v
      } \notag\\
    &= -\x + \y \notag
\end{align}
Therefore,
\begin{align}
  \x &= \y \label{eq:tosub}
\end{align}
Now according to Galerkin's method, we take $v = \phi = N_j\phi_j$, which gives
\begin{align}
  \nabla v &= \nabla \phi = \nabla (N_i \phi_i) \notag\\
           &= \sum_i \phi_i \nabla N_i \notag\\
  &= \sum_i \phi_i
    \begin{bmatrix}
      \dx{N_i} \\ \dy{N_i}
    \end{bmatrix} \notag
\end{align}
Therefore, substituting the above into \cref{eq:tosub} gives
\def\x{\ensuremath{\inta{\nabla (N_i)\nabla (N_j)}}}
\def\bi{\ensuremath{\intb{\phin N_i}}}
\begin{align}
  \phi_i \inta{\nabla N_i \nabla N_j} \phi_j &= \intb{\phin N_j \phi_j} \notag\\
  \phi_i K_{ij} \phi_j &= b_i \phi_i \notag\\
  \intertext{Where}
  K_{ij} &= \x = \inta{\dx{N_i}\dx{N_j} + \dy{N_x}\dy{N_j}} \notag\\
  b_i &= \bi \notag
\end{align}
Taking derivatives with respect to $\mathbf{\phi} = [\phi_1, \dotsc, \phi_n]$,
and setting it to 0 gives the finite element equation
\def\tr#1{\ensuremath{{#1}^{\text{T}}}}
\def\p{\mathbf{\phi}}
\begin{align}
  \dX[\p]{\tr{\p} \mathbf{K} \p - \tr{\p}\mathbf{b}} &= 0 \quad \Rightarrow \quad
                                                       \mathbf{K}\p = \mathbf{b}
                                                       \label{eq:fe} \\
  \intertext{Where}
  \mathbf{K}_{ij} &= K_{ij} \quad \mathbf{b}_{i} &= b_i \notag
\end{align}
\def\K{\ensuremath{\mathbf{K}}}
We calculate \K element by element. But before we start, we derive the general
formulae that calculate the ``local stiffness matrix'' $K_{ij} = \x $ on a
rectangular element and a triangular element.

\subsubsection*{Local stiffness matrix for a rectangular element}
For a rectangular element as shown below

\begin{center}
  \begin{tikzpicture}[x=1cm,y=1cm]
  \begin{scope}[very thick]
    \draw (0,0) node[above right] (n1) {1}
    --  +(5,0) node[above right] (n2) {2}
    -- +(5,3) node[above right] (n3) {3}
    -- +(0,3) node[above right] (n4) {4}
    -- cycle;

    \foreach \i in {1,2,3,4}{
      % \node[fill=#2,]
      \fill[black] (n\i.south west) circle (3pt);
    }

    \newdimen\h
    \h=3.5pt
    % Draw the two dimensions
    \def\i{0.5cm}
    \def\j{5cm}

    \draw[thin,<->] (0,-\i) to node[anchor=north] {$L_1$} +(\j,0);
    \draw (0,-\i) -- +(0,\h) -- +(0, -\h)
    (\j,-\i) -- +(0,\h) -- +(0, -\h);

    % y dimensions
    \newcommand{\mydimy}[4]{%
      \draw[thin,<->] (#1,#2) to node[left] {#4} +(0,#3);
      \draw (#1,#2) -- +(-\h,0) -- +(\h,0)
      (#1,#2 + #3) --  +(-\h,0) -- +(\h,0);
    }
    \mydimy{-0.5cm}{0}{3}{$L_{2}$}
  \end{scope}
\end{tikzpicture}
\end{center}

\newcommand{\shpf}{shape functions}
The \shpf{} are
\newcommand{\Ni}[1]{\ensuremath{N_i(x,y)}}
\newcommand{\Nij}[2]{\dX[#2]{N_{#1}}}
\def\g#1{\ensuremath{g_{#1}(y)}}
\def\f#1{\ensuremath{f_{#1}(x)}}
\newcommand{\vg}{
  \begin{bmatrix}
    \g2 \\ \g1
  \end{bmatrix}
}

\begin{align}
  \begin{bmatrix}
    \Ni{4} & \Ni{3}\\
    \Ni{1} & \Ni{2}
  \end{bmatrix} &= \vg
             \begin{bmatrix}
               \f1 & \f2
             \end{bmatrix} \notag\\
  \intertext{Where}
  \begin{bmatrix}
    \f2\\ \f1
  \end{bmatrix}
           &=
             \begin{bmatrix}
               f_2\\f_1
             \end{bmatrix}
           =
                  \begin{bmatrix}
                    \frac{x}{L1} \\
                    1 - \f2
                  \end{bmatrix} \notag\\
  \vg &=
        \begin{bmatrix}
          g_2\\g_1
        \end{bmatrix}
           =
             \begin{bmatrix}
               \frac{y}{L2} \\
               1 - \g2
             \end{bmatrix} \notag
  \intertext{And their derivatives are}
  \begin{bmatrix}
    f_2' \\
    f_1'
  \end{bmatrix}
           &=
             \begin{bmatrix}
               \frac{1}{L1}\\
               - f_2'
             \end{bmatrix} \notag\\
  \begin{bmatrix}
    g_2' \\
    g_1'
  \end{bmatrix}
           &=
             \begin{bmatrix}
               \frac{1}{L2}\\
               - g_2'
             \end{bmatrix} \label{eq:fgvals}
\end{align}
Note that we have arranged the \shpf{} such that their position on the matrix above
are similar to their corresponding nodes on the figure above.

Now we calculate \Nij{i}{x} and \Nij{i}{y}
\newcommand{\Nx}{\ensuremath{\dx{\mathbf{N}}}}
\newcommand{\Ny}{\ensuremath{\dy{\mathbf{N}}}}

\def\mv#1#2{
  \ensuremath{\begin{bmatrix}#1 \\ #2\end{bmatrix}}
}
\def\mh#1#2{
  \ensuremath{\begin{bmatrix}#1 & #2\end{bmatrix}}
}
\def\x{\mv{g_2}{g_1}}
\def\y{
  % \ensuremath{\begin{bmatrix} f_1 & f_2 \end{bmatrix}}
  \ensuremath{\mh{f_1}{f_2}}
}
\begin{align}
  \Nx &=
        \begin{bmatrix}
          \Nij{4}{x} & \Nij{3}{x} \\
          \Nij{1}{x} & \Nij{2}{x} 
        \end{bmatrix}
                       = \x \mh{f_1'}{f_2'} = \x \mh{-f_2'}{f_2'}
                       = f_2' \x \mh{-1}{1}
                       \notag\\
  \Ny &=
        \begin{bmatrix}
          \Nij{4}{y} & \Nij{3}{y} \\
          \Nij{1}{y} & \Nij{2}{y} 
        \end{bmatrix}
                       = \mv{g_2'}{g_1'} \y = \mv{g_2'}{-g_2'} \y
                       = g_2'\mv{1}{-1} \y \notag
\end{align}
\def\Kx{\ensuremath{\mathbf{K_x}}}
\def\Ky{\ensuremath{\mathbf{K_y}}}
\newcommand{\Mij}[1]{
  \ensuremath{
    {\left[ #1 \right]}_{ij}
  }
}
Now er define two matrices \Kx and \Ky as follows:
\newcommand{\intc}[1]{
  \ensuremath{\int_0^{L2} \int_0^{L1} #1 \text{dxdy}}
}
\begin{align}
  \K &= \Kx + \Ky \notag\\
  \intertext{Where}
  \Mij{\Kx} &= \intc{\Nij{i}{x} \Nij{j}{x}} \notag\\
  \Mij{\Ky} &= \intc{\Nij{i}{y} \Nij{j}{y}} \notag
\end{align}
Next we flatten the (derivatives) \shpf{} matrices into column vectors by
defining:
\def\f#1{\ensuremath{N_{#1} &=
    \begin{bmatrix}
      \Nij{1}{#1} &
      \Nij{2}{#1} &
      \Nij{3}{#1} &
      \Nij{4}{#1} 
    \end{bmatrix}
  }
}
\begin{align}
  \f{x} \notag\\
  \f{y} \notag
\end{align}
So, $K_x$ and $K_y$ are
\begin{align}
  K_x &= \intc{\tr{N_x}N_x} \notag\\ 
  K_y &= \intc{\tr{N_y}N_y} \notag
\end{align}
Substituting the values in \cref{eq:fgvals} gives

\newcommand{\G}[1]{\MyGet{#1}}
\begin{align}
  K &= K_x + K_y \label{eq:recK} \\
  K_x &= \frac{1}{6} \frac{L_2}{L_1} \G{rec.Kx} \notag\\ 
  K_y &= \frac{1}{6} \frac{L_1}{L_2} \G{rec.Ky} \notag
\end{align}

\subsubsection*{Local stiffness matrix for a triangular element}
For a triangular element as shown below

\begin{center}
\begin{tikzpicture}[x=1cm,y=1cm]
  \begin{scope}
    \draw (0,0) coordinate (n1)
    --  +(4,-1) coordinate (n2) 
    -- +(3,2) coordinate (n3)
    -- cycle;

    \foreach \i/\p in {1/above left,2/below right,
      3/above left}{
      \node[\p] at (n\i) {$(x_{\i}, y_{\i})$};
      \fill[black] (n\i) circle (3pt);
    }
  \end{scope}
\end{tikzpicture}
\end{center}

The \shpf{} are
\def\N{\ensuremath{\mathbf{N}}}
\def\B{\ensuremath{\mathbf{B}}}
\def\A{\ensuremath{\mathbf{A}}}
\def\vxy{\ensuremath{\begin{bmatrix} 1&x&y \end{bmatrix}}}
\begin{align}
  \N &=
       \begin{bmatrix}
         \Ni{1} & \Ni{2} & \Ni{3} 
       \end{bmatrix} \notag\\
     &= \vxy \B = \vxy
       \begin{bmatrix} r_1 \\ r_2 \\ r_3 \end{bmatrix} \notag\\
  &= r_1 + xr_2 + yr_3
  \intertext{Where}
  r_i &= \text{The $i^{th}$ row of \B} \notag\\
  \B &= \A^{-1} \notag\\
  \A &=
       \begin{bmatrix}
         1 & x_1 & y_1 \\
         1 & x_2 & y_2 \\
         1 & x_3 & y_3 
       \end{bmatrix} \notag
\end{align}
Note that \shpf{} are linear, so their derivatives with respect to $x$ and $y$
are constant, which turns out to be entries in \B.
\def\f#1{
  \ensuremath{
    N_{#1} := \dX[#1]{\N} =
    \tr{
      \begin{bmatrix}
        \dX[#1]{N_1} &
        \dX[#1]{N_2} &
        \dX[#1]{N_3} 
      \end{bmatrix}
    }
      }
    }
\begin{align}
  \f{x} = r_2 \notag\\
  \f{y} = r_3 \notag
\end{align}
Similar to what we have done to for the rectanglar element, we define \Kx and
\Ky as follows:

\newcommand{\intd}[1]{
  \ensuremath{
    \int_{\text{A}}
    #1
    \text{dA}
  }
}
\def\f#1{
  \ensuremath{
    \Mij{K_{#1}} := \intd{\dX[#1]{N_i} \dX[#1]{N_j}}
  }
}
\def\dN#1{
  \ensuremath{\dX[#1]{\N}}
}
\def\dNdN#1{\tr{N_{x}}N_{x}}

\begin{align}
  \f{x} \notag\\
  \f{y}
  \notag
\end{align}
Where integrate over A means ``Integrate over the entire triangular element''.
However, since $N_x$ and $N_y$ are actually constant vector, therefore
\def\ar{ \text{area of triangular element}}
\begin{align}
  K_x  = \intd{\dNdN{x}} = r_2\tr{r_2} \intd{} = \ar \notag\\ 
  K_y = \intd{\dNdN{y}}  = r_2\tr{r_3} \intd{} = \ar \notag\\
  \intertext{So}
  \K = [r_2\tr{r_2} + r_3\tr{r_3}] \times \ar \label{eq:triK}
\end{align}

\subsubsection*{Calculate \K element by element}
\newcommand{\globalK}[1]{
  And its contribution to global stiffness matrix is
\begin{align}
  K_{#1,\text{global}} = \G{K.g.#1} \label{eq:K.g.#1}
\end{align}
}
\newcommand{\rec}[1]{%
  \paragraph{For element #1} applying \cref{eq:recK}
  with $L_1 =
  \G{l1.#1}, L_2 = \G{l2.#1}$ gives
\begin{align}
  K_{x,#1} &= \G{Kx.#1} \notag\\
  K_{y,#1} &= \G{Ky.#1} \notag\\
  K_{#1} &= \G{K.#1} \notag
\end{align}
\globalK{#1}
}
\newcommand{\tri}[1]{
  \paragraph{For element #1} applying \cref{eq:triK} with
  \begin{align*}
    \A = \begin{bmatrix}
      1 & x_1 & y_1 \\
      1 & x_2 & y_2 \\
      1 & x_3 & y_3 
    \end{bmatrix} = \G{A.#1}
  \end{align*} gives
  \begin{align}
    \B &= \A^{-1} = \G{B.#1} \notag\\
    \K &= \left[
         \G{B.r2.#1} \G{B.r2.t.#1}
         +
         \G{B.r3.#1} \G{B.r3.t.#1}
         \right]
         \G{Area.#1} \notag
\end{align}
\globalK{#1}
}
\rec{a}
\rec{b}
\tri{c}
\tri{d}
\tri{e}
\tri{f}
Now summing up \cref{eq:K.g.a,eq:K.g.b,eq:K.g.c,eq:K.g.d,eq:K.g.e,eq:K.g.f}
gives the global stiffness matrix \K
\begin{align}
  K_{\text{global}} &=
                      K_{a,\text{global}}
                      \foreach \i in {b,c,d,e,f}{
                      + 
                      K_{\i,\text{global}}
                      }\notag\\ 
  &= \G{K.g}
\end{align}

To finish \cref{eq:fe}, we apply the BCs, which can be more or less translated
into the following:
\def\DOF{degree of freedom}
\begin{quote}
  On all nodes except for node 3 and 4,
  the values of $\phi$ are known (so  \DOF{}$-7$). However, for each of these
  node (7 nodes), a nonzero value of $b_i = \bi$ is added for them.
\end{quote}
So
\def\au{\ensuremath{au_0}}
\begin{align}
  \mathbf{b} &=
               \tr{\begin{bmatrix}
                 b_1 & b_2 & 0&0&b_5&b_6&b_7&b_8&b_9
               \end{bmatrix}}
                                                  \notag\\
  \intertext{And}
  \mathbf{\phi}
             &=
                  \tr{\begin{bmatrix}
                      b_1 & b_2 & 0&0&b_5&b_6&b_7&b_8&b_9
                    \end{bmatrix}} \notag\\
             &=
               \tr{\begin{bmatrix}
                   \phi_1 , \dotsc \phi_9
                 \end{bmatrix}} \notag\\
             &=
               \tr{\begin{bmatrix}
                   0&0&\phi_3&\phi_4&0&0&\au&\au&\au
                 \end{bmatrix}}
\end{align}
So the system is therefore
\def\bn{\ensuremath{
    \begin{bmatrix}
      b_1 \\ b_2 \\ 0 \\ 0 \\ b_5 \\ b_6 \\ b_7 \\ b_8 \\ b_9 
    \end{bmatrix}}}
\def\phin{\ensuremath{
    \begin{bmatrix}
      0\\0\\\phi_3\\\phi_4\\0\\0\\ \au \\ \au \\ \au
    \end{bmatrix}}}
\begin{align*}
  \K \mathbf{\phi} = \mathbf{b} \notag\\
  \G{K.g} \phin = \bn \notag
\end{align*}
We first solve for $\phi_2$ and $\phi_3$
\def\one{\ensuremath{
    \begin{bmatrix} 1\\1\\1
    \end{bmatrix}}}
\def\pp{\ensuremath{
    \begin{bmatrix} \phi_3 \\ \phi_4
    \end{bmatrix}}}
\begin{align}
  \G{K.g.sm} \phin &= \bn \notag\\
  \G{K.g.m1} \pp &= - \G{K.g.m2} \one \au \notag\\
  \G{K.g.m1}^{-1} \pp &= \G{K.g.mi.inv} \notag\\
  -\au \G{K.g.m2} \one &= \au \G{K.g.m3} \notag\\
  \pp &= \au \begin{bmatrix} \G{phi3}\\\G{phi4} \end{bmatrix} \notag
\end{align}

And now $\mathbf{b}$ can be calculated as
\begin{align}
  \mathbf{b} = \au \G{K.g}
  \begin{bmatrix}
    0\\0\\\G{phi3}\\\G{phi4}\\0\\0\\ \au \\ \au \\ \au
  \end{bmatrix}
  = \au \G{b.n}
\end{align}

\subsubsection*{Remarks}
And the contour is shown below

% Created by tikzDevice version 0.12.3.1 on 2022-03-13 23:33:43
% !TEX encoding = UTF-8 Unicode
\begin{tikzpicture}[x=1pt,y=1pt]
\definecolor{fillColor}{RGB}{255,255,255}
\path[use as bounding box,fill=fillColor,fill opacity=0.00] (0,0) rectangle (867.24,361.35);
\begin{scope}
\path[clip] (  0.00,  0.00) rectangle (867.24,361.35);
\definecolor{drawColor}{RGB}{255,255,255}
\definecolor{fillColor}{RGB}{255,255,255}

\path[draw=drawColor,line width= 0.6pt,line join=round,line cap=round,fill=fillColor] (  0.00,  0.00) rectangle (867.24,361.35);
\end{scope}
\begin{scope}
\path[clip] ( 14.85, 18.22) rectangle (804.15,355.85);
\definecolor{fillColor}{gray}{0.92}

\path[fill=fillColor] ( 14.85, 18.22) rectangle (804.15,355.85);
\definecolor{drawColor}{RGB}{255,255,255}

\path[draw=drawColor,line width= 0.3pt,line join=round] ( 14.85, 42.09) --
	(804.15, 42.09);

\path[draw=drawColor,line width= 0.3pt,line join=round] ( 14.85,161.46) --
	(804.15,161.46);

\path[draw=drawColor,line width= 0.3pt,line join=round] ( 14.85,280.82) --
	(804.15,280.82);

\path[draw=drawColor,line width= 0.3pt,line join=round] (140.42, 18.22) --
	(140.42,355.85);

\path[draw=drawColor,line width= 0.3pt,line join=round] (319.80, 18.22) --
	(319.80,355.85);

\path[draw=drawColor,line width= 0.3pt,line join=round] (499.19, 18.22) --
	(499.19,355.85);

\path[draw=drawColor,line width= 0.3pt,line join=round] (678.58, 18.22) --
	(678.58,355.85);

\path[draw=drawColor,line width= 0.6pt,line join=round] ( 14.85,101.78) --
	(804.15,101.78);

\path[draw=drawColor,line width= 0.6pt,line join=round] ( 14.85,221.14) --
	(804.15,221.14);

\path[draw=drawColor,line width= 0.6pt,line join=round] ( 14.85,340.50) --
	(804.15,340.50);

\path[draw=drawColor,line width= 0.6pt,line join=round] ( 50.73, 18.22) --
	( 50.73,355.85);

\path[draw=drawColor,line width= 0.6pt,line join=round] (230.11, 18.22) --
	(230.11,355.85);

\path[draw=drawColor,line width= 0.6pt,line join=round] (409.50, 18.22) --
	(409.50,355.85);

\path[draw=drawColor,line width= 0.6pt,line join=round] (588.88, 18.22) --
	(588.88,355.85);

\path[draw=drawColor,line width= 0.6pt,line join=round] (768.27, 18.22) --
	(768.27,355.85);
\definecolor{drawColor}{RGB}{35,75,110}
\definecolor{fillColor}{RGB}{35,75,110}

\path[draw=drawColor,line width= 0.4pt,line join=round,line cap=round,fill=fillColor] ( 50.73, 33.57) circle (  1.96);
\definecolor{drawColor}{RGB}{36,77,113}
\definecolor{fillColor}{RGB}{36,77,113}

\path[draw=drawColor,line width= 0.4pt,line join=round,line cap=round,fill=fillColor] (153.23, 33.57) circle (  1.96);
\definecolor{drawColor}{RGB}{38,80,117}
\definecolor{fillColor}{RGB}{38,80,117}

\path[draw=drawColor,line width= 0.4pt,line join=round,line cap=round,fill=fillColor] (255.74, 33.57) circle (  1.96);
\definecolor{drawColor}{RGB}{39,82,120}
\definecolor{fillColor}{RGB}{39,82,120}

\path[draw=drawColor,line width= 0.4pt,line join=round,line cap=round,fill=fillColor] (358.24, 33.57) circle (  1.96);
\definecolor{drawColor}{RGB}{30,64,95}
\definecolor{fillColor}{RGB}{30,64,95}

\path[draw=drawColor,line width= 0.4pt,line join=round,line cap=round,fill=fillColor] (460.75, 33.57) circle (  1.96);
\definecolor{drawColor}{RGB}{30,65,97}
\definecolor{fillColor}{RGB}{30,65,97}

\path[draw=drawColor,line width= 0.4pt,line join=round,line cap=round,fill=fillColor] ( 50.73, 84.72) circle (  1.96);
\definecolor{drawColor}{RGB}{31,66,98}
\definecolor{fillColor}{RGB}{31,66,98}

\path[draw=drawColor,line width= 0.4pt,line join=round,line cap=round,fill=fillColor] (153.23, 84.72) circle (  1.96);
\definecolor{drawColor}{RGB}{31,67,99}
\definecolor{fillColor}{RGB}{31,67,99}

\path[draw=drawColor,line width= 0.4pt,line join=round,line cap=round,fill=fillColor] (255.74, 84.72) circle (  1.96);
\definecolor{drawColor}{RGB}{31,67,100}
\definecolor{fillColor}{RGB}{31,67,100}

\path[draw=drawColor,line width= 0.4pt,line join=round,line cap=round,fill=fillColor] (358.24, 84.72) circle (  1.96);
\definecolor{drawColor}{RGB}{34,73,108}
\definecolor{fillColor}{RGB}{34,73,108}

\path[draw=drawColor,line width= 0.4pt,line join=round,line cap=round,fill=fillColor] (460.75, 84.72) circle (  1.96);
\definecolor{drawColor}{RGB}{56,116,165}
\definecolor{fillColor}{RGB}{56,116,165}

\path[draw=drawColor,line width= 0.4pt,line join=round,line cap=round,fill=fillColor] ( 50.73,135.88) circle (  1.96);
\definecolor{drawColor}{RGB}{57,119,169}
\definecolor{fillColor}{RGB}{57,119,169}

\path[draw=drawColor,line width= 0.4pt,line join=round,line cap=round,fill=fillColor] (153.23,135.88) circle (  1.96);
\definecolor{drawColor}{RGB}{59,122,173}
\definecolor{fillColor}{RGB}{59,122,173}

\path[draw=drawColor,line width= 0.4pt,line join=round,line cap=round,fill=fillColor] (255.74,135.88) circle (  1.96);
\definecolor{drawColor}{RGB}{60,125,178}
\definecolor{fillColor}{RGB}{60,125,178}

\path[draw=drawColor,line width= 0.4pt,line join=round,line cap=round,fill=fillColor] (358.24,135.88) circle (  1.96);
\definecolor{drawColor}{RGB}{42,89,128}
\definecolor{fillColor}{RGB}{42,89,128}

\path[draw=drawColor,line width= 0.4pt,line join=round,line cap=round,fill=fillColor] (460.75,135.88) circle (  1.96);
\definecolor{drawColor}{RGB}{34,72,106}
\definecolor{fillColor}{RGB}{34,72,106}

\path[draw=drawColor,line width= 0.4pt,line join=round,line cap=round,fill=fillColor] (563.26,135.88) circle (  1.96);
\definecolor{drawColor}{RGB}{46,97,139}
\definecolor{fillColor}{RGB}{46,97,139}

\path[draw=drawColor,line width= 0.4pt,line join=round,line cap=round,fill=fillColor] ( 50.73,187.04) circle (  1.96);
\definecolor{drawColor}{RGB}{48,101,145}
\definecolor{fillColor}{RGB}{48,101,145}

\path[draw=drawColor,line width= 0.4pt,line join=round,line cap=round,fill=fillColor] (153.23,187.04) circle (  1.96);
\definecolor{drawColor}{RGB}{50,105,150}
\definecolor{fillColor}{RGB}{50,105,150}

\path[draw=drawColor,line width= 0.4pt,line join=round,line cap=round,fill=fillColor] (255.74,187.04) circle (  1.96);
\definecolor{drawColor}{RGB}{52,109,155}
\definecolor{fillColor}{RGB}{52,109,155}

\path[draw=drawColor,line width= 0.4pt,line join=round,line cap=round,fill=fillColor] (358.24,187.04) circle (  1.96);
\definecolor{drawColor}{RGB}{51,108,154}
\definecolor{fillColor}{RGB}{51,108,154}

\path[draw=drawColor,line width= 0.4pt,line join=round,line cap=round,fill=fillColor] (460.75,187.04) circle (  1.96);
\definecolor{drawColor}{RGB}{42,89,129}
\definecolor{fillColor}{RGB}{42,89,129}

\path[draw=drawColor,line width= 0.4pt,line join=round,line cap=round,fill=fillColor] (563.26,187.04) circle (  1.96);
\definecolor{drawColor}{RGB}{33,70,103}
\definecolor{fillColor}{RGB}{33,70,103}

\path[draw=drawColor,line width= 0.4pt,line join=round,line cap=round,fill=fillColor] (665.76,187.04) circle (  1.96);
\definecolor{drawColor}{RGB}{37,78,114}
\definecolor{fillColor}{RGB}{37,78,114}

\path[draw=drawColor,line width= 0.4pt,line join=round,line cap=round,fill=fillColor] ( 50.73,238.19) circle (  1.96);
\definecolor{drawColor}{RGB}{39,83,121}
\definecolor{fillColor}{RGB}{39,83,121}

\path[draw=drawColor,line width= 0.4pt,line join=round,line cap=round,fill=fillColor] (153.23,238.19) circle (  1.96);
\definecolor{drawColor}{RGB}{42,88,127}
\definecolor{fillColor}{RGB}{42,88,127}

\path[draw=drawColor,line width= 0.4pt,line join=round,line cap=round,fill=fillColor] (255.74,238.19) circle (  1.96);
\definecolor{drawColor}{RGB}{44,93,134}
\definecolor{fillColor}{RGB}{44,93,134}

\path[draw=drawColor,line width= 0.4pt,line join=round,line cap=round,fill=fillColor] (358.24,238.19) circle (  1.96);
\definecolor{drawColor}{RGB}{61,127,180}
\definecolor{fillColor}{RGB}{61,127,180}

\path[draw=drawColor,line width= 0.4pt,line join=round,line cap=round,fill=fillColor] (460.75,238.19) circle (  1.96);
\definecolor{drawColor}{RGB}{51,107,153}
\definecolor{fillColor}{RGB}{51,107,153}

\path[draw=drawColor,line width= 0.4pt,line join=round,line cap=round,fill=fillColor] (563.26,238.19) circle (  1.96);
\definecolor{drawColor}{RGB}{41,87,126}
\definecolor{fillColor}{RGB}{41,87,126}

\path[draw=drawColor,line width= 0.4pt,line join=round,line cap=round,fill=fillColor] (665.76,238.19) circle (  1.96);
\definecolor{drawColor}{RGB}{36,77,113}
\definecolor{fillColor}{RGB}{36,77,113}

\path[draw=drawColor,line width= 0.4pt,line join=round,line cap=round,fill=fillColor] (768.27,238.19) circle (  1.96);
\definecolor{drawColor}{RGB}{28,60,90}
\definecolor{fillColor}{RGB}{28,60,90}

\path[draw=drawColor,line width= 0.4pt,line join=round,line cap=round,fill=fillColor] ( 50.73,289.35) circle (  1.96);
\definecolor{drawColor}{RGB}{30,66,98}
\definecolor{fillColor}{RGB}{30,66,98}

\path[draw=drawColor,line width= 0.4pt,line join=round,line cap=round,fill=fillColor] (153.23,289.35) circle (  1.96);
\definecolor{drawColor}{RGB}{33,71,105}
\definecolor{fillColor}{RGB}{33,71,105}

\path[draw=drawColor,line width= 0.4pt,line join=round,line cap=round,fill=fillColor] (255.74,289.35) circle (  1.96);
\definecolor{drawColor}{RGB}{36,77,113}
\definecolor{fillColor}{RGB}{36,77,113}

\path[draw=drawColor,line width= 0.4pt,line join=round,line cap=round,fill=fillColor] (358.24,289.35) circle (  1.96);
\definecolor{drawColor}{RGB}{71,147,206}
\definecolor{fillColor}{RGB}{71,147,206}

\path[draw=drawColor,line width= 0.4pt,line join=round,line cap=round,fill=fillColor] (460.75,289.35) circle (  1.96);
\definecolor{drawColor}{RGB}{60,125,178}
\definecolor{fillColor}{RGB}{60,125,178}

\path[draw=drawColor,line width= 0.4pt,line join=round,line cap=round,fill=fillColor] (563.26,289.35) circle (  1.96);

\path[draw=drawColor,line width= 0.4pt,line join=round,line cap=round,fill=fillColor] (665.76,289.35) circle (  1.96);

\path[draw=drawColor,line width= 0.4pt,line join=round,line cap=round,fill=fillColor] (768.27,289.35) circle (  1.96);
\definecolor{drawColor}{RGB}{19,43,67}
\definecolor{fillColor}{RGB}{19,43,67}

\path[draw=drawColor,line width= 0.4pt,line join=round,line cap=round,fill=fillColor] ( 50.73,340.50) circle (  1.96);
\definecolor{drawColor}{RGB}{22,49,75}
\definecolor{fillColor}{RGB}{22,49,75}

\path[draw=drawColor,line width= 0.4pt,line join=round,line cap=round,fill=fillColor] (153.23,340.50) circle (  1.96);
\definecolor{drawColor}{RGB}{25,56,84}
\definecolor{fillColor}{RGB}{25,56,84}

\path[draw=drawColor,line width= 0.4pt,line join=round,line cap=round,fill=fillColor] (255.74,340.50) circle (  1.96);
\definecolor{drawColor}{RGB}{29,62,93}
\definecolor{fillColor}{RGB}{29,62,93}

\path[draw=drawColor,line width= 0.4pt,line join=round,line cap=round,fill=fillColor] (358.24,340.50) circle (  1.96);
\definecolor{drawColor}{RGB}{86,177,247}
\definecolor{fillColor}{RGB}{86,177,247}

\path[draw=drawColor,line width= 0.4pt,line join=round,line cap=round,fill=fillColor] (460.75,340.50) circle (  1.96);

\path[draw=drawColor,line width= 0.4pt,line join=round,line cap=round,fill=fillColor] (563.26,340.50) circle (  1.96);

\path[draw=drawColor,line width= 0.4pt,line join=round,line cap=round,fill=fillColor] (665.76,340.50) circle (  1.96);

\path[draw=drawColor,line width= 0.4pt,line join=round,line cap=round,fill=fillColor] (768.27,340.50) circle (  1.96);
\end{scope}
\begin{scope}
\path[clip] (  0.00,  0.00) rectangle (867.24,361.35);
\definecolor{drawColor}{gray}{0.30}

\node[text=drawColor,anchor=base east,inner sep=0pt, outer sep=0pt, scale=  0.88] at (  9.90, 98.75) {2};

\node[text=drawColor,anchor=base east,inner sep=0pt, outer sep=0pt, scale=  0.88] at (  9.90,218.11) {4};

\node[text=drawColor,anchor=base east,inner sep=0pt, outer sep=0pt, scale=  0.88] at (  9.90,337.47) {6};
\end{scope}
\begin{scope}
\path[clip] (  0.00,  0.00) rectangle (867.24,361.35);
\definecolor{drawColor}{gray}{0.20}

\path[draw=drawColor,line width= 0.6pt,line join=round] ( 12.10,101.78) --
	( 14.85,101.78);

\path[draw=drawColor,line width= 0.6pt,line join=round] ( 12.10,221.14) --
	( 14.85,221.14);

\path[draw=drawColor,line width= 0.6pt,line join=round] ( 12.10,340.50) --
	( 14.85,340.50);
\end{scope}
\begin{scope}
\path[clip] (  0.00,  0.00) rectangle (867.24,361.35);
\definecolor{drawColor}{gray}{0.20}

\path[draw=drawColor,line width= 0.6pt,line join=round] ( 50.73, 15.47) --
	( 50.73, 18.22);

\path[draw=drawColor,line width= 0.6pt,line join=round] (230.11, 15.47) --
	(230.11, 18.22);

\path[draw=drawColor,line width= 0.6pt,line join=round] (409.50, 15.47) --
	(409.50, 18.22);

\path[draw=drawColor,line width= 0.6pt,line join=round] (588.88, 15.47) --
	(588.88, 18.22);

\path[draw=drawColor,line width= 0.6pt,line join=round] (768.27, 15.47) --
	(768.27, 18.22);
\end{scope}
\begin{scope}
\path[clip] (  0.00,  0.00) rectangle (867.24,361.35);
\definecolor{drawColor}{gray}{0.30}

\node[text=drawColor,anchor=base,inner sep=0pt, outer sep=0pt, scale=  0.88] at ( 50.73,  7.21) {0.0};

\node[text=drawColor,anchor=base,inner sep=0pt, outer sep=0pt, scale=  0.88] at (230.11,  7.21) {2.5};

\node[text=drawColor,anchor=base,inner sep=0pt, outer sep=0pt, scale=  0.88] at (409.50,  7.21) {5.0};

\node[text=drawColor,anchor=base,inner sep=0pt, outer sep=0pt, scale=  0.88] at (588.88,  7.21) {7.5};

\node[text=drawColor,anchor=base,inner sep=0pt, outer sep=0pt, scale=  0.88] at (768.27,  7.21) {10.0};
\end{scope}
\begin{scope}
\path[clip] (  0.00,  0.00) rectangle (867.24,361.35);
\definecolor{fillColor}{RGB}{255,255,255}

\path[fill=fillColor] (815.15,137.79) rectangle (861.74,236.28);
\end{scope}
\begin{scope}
\path[clip] (  0.00,  0.00) rectangle (867.24,361.35);
\node[inner sep=0pt,outer sep=0pt,anchor=south west,rotate=  0.00] at (820.65, 143.29) {
	\pgfimage[width= 14.45pt,height= 72.27pt,interpolate=true]{p_ras1}};
\end{scope}
\begin{scope}
\path[clip] (  0.00,  0.00) rectangle (867.24,361.35);
\definecolor{drawColor}{RGB}{0,0,0}

\node[text=drawColor,anchor=base west,inner sep=0pt, outer sep=0pt, scale=  0.88] at (840.60,152.08) {0.00};

\node[text=drawColor,anchor=base west,inner sep=0pt, outer sep=0pt, scale=  0.88] at (840.60,167.16) {0.25};

\node[text=drawColor,anchor=base west,inner sep=0pt, outer sep=0pt, scale=  0.88] at (840.60,182.25) {0.50};

\node[text=drawColor,anchor=base west,inner sep=0pt, outer sep=0pt, scale=  0.88] at (840.60,197.33) {0.75};

\node[text=drawColor,anchor=base west,inner sep=0pt, outer sep=0pt, scale=  0.88] at (840.60,212.41) {1.00};
\end{scope}
\begin{scope}
\path[clip] (  0.00,  0.00) rectangle (867.24,361.35);
\definecolor{drawColor}{RGB}{0,0,0}

\node[text=drawColor,anchor=base west,inner sep=0pt, outer sep=0pt, scale=  1.10] at (820.65,222.13) {$\phi$};
\end{scope}
\begin{scope}
\path[clip] (  0.00,  0.00) rectangle (867.24,361.35);
\definecolor{drawColor}{RGB}{255,255,255}

\path[draw=drawColor,line width= 0.2pt,line join=round] (820.65,155.11) -- (823.54,155.11);

\path[draw=drawColor,line width= 0.2pt,line join=round] (820.65,170.19) -- (823.54,170.19);

\path[draw=drawColor,line width= 0.2pt,line join=round] (820.65,185.28) -- (823.54,185.28);

\path[draw=drawColor,line width= 0.2pt,line join=round] (820.65,200.36) -- (823.54,200.36);

\path[draw=drawColor,line width= 0.2pt,line join=round] (820.65,215.44) -- (823.54,215.44);

\path[draw=drawColor,line width= 0.2pt,line join=round] (832.21,155.11) -- (835.10,155.11);

\path[draw=drawColor,line width= 0.2pt,line join=round] (832.21,170.19) -- (835.10,170.19);

\path[draw=drawColor,line width= 0.2pt,line join=round] (832.21,185.28) -- (835.10,185.28);

\path[draw=drawColor,line width= 0.2pt,line join=round] (832.21,200.36) -- (835.10,200.36);

\path[draw=drawColor,line width= 0.2pt,line join=round] (832.21,215.44) -- (835.10,215.44);
\end{scope}
\end{tikzpicture}


We see that the value of $\phi$ gradually increase from bottom ($\phi=0$) to
top ($\phi=1$). More elements shall be used for an increased accuracy.
