\message{ !name(p2.tex)}
\message{ !name(p2.tex) !offset(-2) }
\newcommand{\dXX}[2][x]{
  \ensuremath{\frac{\partial^2 #2}{\partial #1^2}}
}
\newcommand{\dX}[2][x]{
  \ensuremath{\frac{\partial #2}{\partial #1}}
}
\newcommand{\dxx}[1]{\dXX{#1}}
\newcommand{\dyy}[1]{\dXX[y]{#1}}
\newcommand{\dx}[1]{\dX{#1}}
\newcommand{\dy}[1]{\dX[y]{#1}}
\newcommand{\phin}{\ensuremath{
    \dX[\mathbf{n}]{\phi} 
  }}
\begin{questionbox}
  The function $\phi(x,y)$ is governed by
\begin{align}
\dxx{\phi} + \dyy{\phi} = 0 \label{eq:poi}
\end{align}
The mesh is shown in the following figure:

\newtcbox{bLabBox}[1][red]{
  on line,arc=0pt,colback=#1!10!white,
  colframe=#1!50!black,
  boxsep=0pt,left=1pt,right=1pt,top=2pt,
  bottom=2pt,boxrule=0pt,bottomrule=1pt,toprule=1pt
}
\newtcbox{nLabBox}[1][red]{
  on line,top=2pt,bottom=2pt,
  left=3pt,right=3pt,
  colframe=#1,
  colback=#1!10!white
}
\begin{tikzpicture}[x=1cm,y=1cm]
  \newcommand{\bLab}[4][(0,0)]{
    % \node[fill=#2,text=white,#3] at #1 {#4};
    \node[#3] at #1 {
      \bLabBox[#2]{#4}
    };
  }

  \begin{scope}[very thick]
    \draw[style=help lines,step=1cm] (0,0) grid (10,6);
    % \tikzstyle{every node}=[draw,circle]

    \draw[draw=\mycola] (0,0) -- (0,6);
    \bLab[(0-0.1,3)]{\mycola}{left}{1}

    \draw[draw=\mycolb] (0,6) -- (10,6);
    \bLab[(4.5,6.1)]{\mycolb}{above}{2}

    \draw[draw=\mycolc] (10,6) -- (10,4);
    \bLab[(10.1,5)]{\mycolc}{right}{3}

    \draw[draw=\mycola] (10,4) -- (7,2) -- (5,0);
    \bLab[(7-0.2,1.2)]{\mycola}{right}{4}

    \draw[draw=\mycolb] (5,0) -- (0,0);
    \bLab[(2.5,-0.1)]{\mycolb}{below}{5}
  \end{scope}

  \begin{scope}
    % draw the mesh
    \draw (0,2) -- +(7,0)
    (5,0) -- ++(0,6)
     -- +(2,-4)
     -- +(5,-2)
     -- +(0,0);

     \foreach \x/\y/\l in {
       0/0/1,5/0/2,0/2/3,5/2/4,
       7/2/5,10/4/6,0/6/7,5/6/8,
       10/6/9}{
       \fill[black] (\x,\y) circle (3pt);
       \node[above right] at (\x,\y) {\l};
     }

     \newcommand{\nLab}[3][(0,0)]{
       % \node[fill=#2,text=white,rectangle] at #1 {#3};
       \node at #1 {\nLabBox[#2]{#3}};
     }

     \foreach \x/\y/\l/\c in {
       2.5/1/a/\mycola,2.5/4/b/\mycolb,
       6/1.5/c/\mycolc,6/3/d/\mycola,
       7/4/e/\mycolb,8/5.5/f/\mycolc
     }{\nLab[(\x,\y)]{\c}{\l}}
  \end{scope}
\end{tikzpicture}

\MySet{bd.col.1}{\mycola}
\MySet{bd.col.2}{\mycolb}
\MySet{bd.col.3}{\mycolc}
\MySet{bd.col.4}{\mycola}
\MySet{bd.col.5}{\mycolb}
\foreach \n/\i in {
  1/\mycola,
  2/\mycolb,
  3/\mycolc,
  4/\mycola,
  5/\mycolb}{
  \MySet{bd.col.\n}{\i}
}

\newcommand{\bd}[1]{\bLabBox[\MyGet{bd.col.#1}]{#1}}
\newcommand{\el}[1]{\nLabBox[\MyGet{el.col.#1}]{#1}}

% \foreach \n/\i in {
%   a/\mycola,
%   b/\mycolb,
%   c/\mycolc,
%   d/\mycola,
%   e/\mycolb,
%   f/\mycolc}{
%   \MySet{el.col.\n}{\i}
% }
% Don't know why it dosn't work   
\MySet{el.col.a}{\mycola}
\MySet{el.col.b}{\mycolb}
\MySet{el.col.c}{\mycolc}
\MySet{el.col.d}{\mycola}
\MySet{el.col.e}{\mycolb}
\MySet{el.col.f}{\mycolc}

Where \bd{1} to \bd{5} are boundaries,
and \el{a} to \el{f} are elements.

And the BCs are:
\def\f#1{\mbox{Along \bd{#1}: }}
\begin{align}
    \f{1} & \phin = 0 &  \f{2} & \phi = au_0 \notag\\
    \f{3} & \phin = 0 & \f{4} & \phi = 0 \notag\\ 
    \f{5} & \phi = 0 && \notag
\end{align}
\end{questionbox}

\subsection*{Answer}\label{sec:q2}
\subsubsection*{Weak formulation}
Let $v$ be an arbitrary test function, the weak formulation of \cref{eq:poi} is
then
\newcommand{\inta}[1]{
  \ensuremath{\int_{\text{Domain}} #1 \text{dA}}
}
\newcommand{\intb}[1]{
  \ensuremath{\int_{\text{Boundary}} #1 \text{ds}}
}

\def\x{
  \ensuremath{\inta{\nabla \phi \bullet \nabla v}}
}
\def\y{
  \ensuremath{\intb{\phin v}}
}


\begin{align}
  0 &= \inta{(\dxx{\phi} + \dyy{\phi}) v} \notag\\
  \intertext{By Gauss-Green}
    &= \inta{-\dx{\phi}\dx{v} - \dy{\phi}\dy{v}} +
      \intb{
      (\dx{\phi}\dx{n} + \dy{\phi}\dy{n})v
      } \notag\\
    &= -\x + \y \notag
\end{align}
Therefore,
\begin{align}
  \x &= \y \label{eq:tosub}
\end{align}
Now according to Galerkin's method, we take $v = \phi = N_j\phi_j$, which gives
\begin{align}
  \nabla v &= \nabla \phi = \nabla (N_i \phi_i) \notag\\
           &= \sum_i \phi_i \nabla N_i \notag\\
  &= \sum_i \phi_i
    \begin{bmatrix}
      \dx{N_i} \\ \dy{N_i}
    \end{bmatrix} \notag
\end{align}
Therefore, substituting the above into \cref{eq:tosub} gives
\def\x{\ensuremath{\inta{\nabla (N_i)\nabla (N_j)}}}
\def\y{\ensuremath{\intb{\phin N_i}}}
\begin{align}
  \phi_i \inta{\nabla N_i \nabla N_j} \phi_j &= \intb{\phin N_j \phi_j} \notag\\
  \phi_i K_{ij} \phi_j &= b_i \phi_i \notag\\
  \intertext{Where}
  K_{ij} &= \x = \inta{\dx{N_i}\dx{N_j} + \dy{N_x}\dy{N_j}} \notag\\
  b_i &= \y \notag
\end{align}
Taking derivatives with respect to $\mathbf{\phi} = [\phi_1, \dotsc, \phi_n]$,
and setting it to 0 gives the finite element equation
\def\tr#1{\ensuremath{{#1}^{\text{T}}}}
\def\p{\mathbf{\phi}}
\begin{align}
  \dX[\p]{\tr{\p} \mathbf{K} \p - \tr{\p}\mathbf{b}} &= 0 \quad \Rightarrow \quad
                                                       \mathbf{K}\p = \mathbf{b}
                                                       \label{eq:fe} \\
  \intertext{Where}
  \mathbf{K}_{ij} &= K_{ij} \quad \mathbf{b}_{i} &= b_i \notag
\end{align}
\def\K{\ensuremath{\mathbf{K}}}
We calculate \K element by element. But before we start, we derive the general
formulae that calculate the ``local stiffness matrix'' $K_{ij} = \x $ on a
rectangular element and a triangular element.

\subsubsection*{Local stiffness matrix for a rectangular element}
For a rectangular element as shown below

\begin{center}
  \begin{tikzpicture}[x=1cm,y=1cm]
  \begin{scope}[very thick]
    \draw (0,0) node[above right] (n1) {1}
    --  +(5,0) node[above right] (n2) {2}
    -- +(5,3) node[above right] (n3) {3}
    -- +(0,3) node[above right] (n4) {4}
    -- cycle;

    \foreach \i in {1,2,3,4}{
      % \node[fill=#2,]
      \fill[black] (n\i.south west) circle (3pt);
    }

    \newdimen\h
    \h=3.5pt
    % Draw the two dimensions
    \def\i{0.5cm}
    \def\j{5cm}

    \draw[thin,<->] (0,-\i) to node[anchor=north] {$L_1$} +(\j,0);
    \draw (0,-\i) -- +(0,\h) -- +(0, -\h)
    (\j,-\i) -- +(0,\h) -- +(0, -\h);

    % y dimensions
    \newcommand{\mydimy}[4]{%
      \draw[thin,<->] (#1,#2) to node[left] {#4} +(0,#3);
      \draw (#1,#2) -- +(-\h,0) -- +(\h,0)
      (#1,#2 + #3) --  +(-\h,0) -- +(\h,0);
    }
    \mydimy{-0.5cm}{0}{3}{$L_{2}$}
  \end{scope}
\end{tikzpicture}
\end{center}

\newcommand{\shpf}{shape functions}
The \shpf{} are
\newcommand{\Ni}[1]{\ensuremath{N_i(x,y)}}
\newcommand{\Nij}[2]{\dX[#2]{#1}}
\def\g#1{\ensuremath{g_{#1}(y)}}
\def\f#1{\ensuremath{f_{#1}(x)}}
\newcommand{\vg}{
  \begin{bmatrix}
    \g2 \\ \g1
  \end{bmatrix}
}

\begin{align}
  \begin{bmatrix}
    \Ni{4} & \Ni{3}\\
    \Ni{1} & \Ni{2}
  \end{bmatrix} &= \vg
             \begin{bmatrix}
               \f1 & \f2
             \end{bmatrix} \notag\\
  \intertext{Where}
  \begin{bmatrix}
    \f2\\ \f1
  \end{bmatrix}
           &=
             \begin{bmatrix}
               f_2\\f_1
             \end{bmatrix}
           =
                  \begin{bmatrix}
                    \frac{x}{L1} \\
                    1 - \f2
                  \end{bmatrix} \notag\\
  \vg &=
        \begin{bmatrix}
          g_2\\g_1
        \end{bmatrix}
           =
             \begin{bmatrix}
               \frac{y}{L2} \\
               1 - \g2
             \end{bmatrix} \notag
  \intertext{And their derivatives are}
  \begin{bmatrix}
    f_2' \\
    f_1'
  \end{bmatrix}
           &=
             \begin{bmatrix}
               \frac{1}{L1}\\
               - f_2'
             \end{bmatrix} \notag\\
  
  \begin{bmatrix}
    g_2' \\
    g_1'
  \end{bmatrix}
           &=
             \begin{bmatrix}
               \frac{1}{L2}\\
               - g_2'
             \end{bmatrix} \label{eq:fgvals}
\end{align}
Note that we have arranged the \shpf{} such that their position on the matrix above
are similar to their corresponding nodes on the figure above.

Now we calculate \N{i}{x} and \N{i}{y}
\newcommand{\Nx}{\ensuremath{\dx{\mathbf{N}}}}
\newcommand{\Ny}{\ensuremath{\dy{\mathbf{N}}}}

\def\x{
  \ensuremath{\begin{bmatrix} g_2 \\ g_1 \end{bmatrix}}}
\def\y{
  \ensuremath{\begin{bmatrix} f_1 & f_2 \end{bmatrix}}}
\begin{align}
  \Nx &=
        \begin{bmatrix}
          \Nij{4}{x} & \Nij{3}{x} \\
          \Nij{1}{x} & \Nij{2}{x} 
        \end{bmatrix} = \x 
\end{align}
\message{ !name(p2.tex) !offset(-226) }
