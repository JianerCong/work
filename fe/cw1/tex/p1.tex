\newcommand{\lhs}{%
  \ensuremath{-(x T')' + \alpha (T - T_{_{\infty}})}
}

\begin{questionbox}
  The temperature distribution $T$ is governed by

  \begin{align}
    \lhs & = 0,
    & (0 \leq x \leq L) \label{eq:base}
  \end{align}
  Where BCs are
  \begin{align}
    {[xT']}_{x=0} &= 0 \label{eq:bc1}\\
    T(L) &= T_0 \label{eq:bc2}
  \end{align}
  And the constants are $L = 4, T_0 = 250, T_{\infty} = 75, \alpha = 0.4168$.
\end{questionbox}

\subsection*{Answer}\label{sec:q1}

\subsubsection*{Applying variational principle}
Applying variational principle, we put the LHS of \cref{eq:base} into the
integral with $\delta T$
\renewcommand{\i}[1]{\ensuremath{\int_0^L #1 dx}}
\begin{align}
  \i{[\lhs] \delta T} &=0 \notag\\
                      &= \cla{-\i{(xT')' \delta T}}
                        +\clb{\i{\alpha T \delta T}}
                        -\clc{\i{\alpha T_{\infty} \delta T}} \label{eq:base2}\\
                      &:= \cla{[1]} + \clb{[2]} + \clc{[3]} \notag
\end{align}

\subsubsection*{Term $\cla{[1]}$}
For $\cla{[1]}$, we integrate by part:
\def\bdr{\cld{\ensuremath{{[xT' \delta T]}_0^L}}}
\def\Bdr{\cld{\ensuremath{{[xT' T]}_0^L}}} %bdr without $\delta$
\begin{align*}
  \cla{[1]} &= -(\bdr - \i{\delta T' x T'}) \\
            &= -(\cdots - \i{x \delta(\frac{1}{2} T'^2) }) \\
            &= -(\cdots - \i{\delta(x\frac{1}{2} T'^2)})
\end{align*}
The term \bdr{} will form the \emph{boundary vector}.

Therefore, $\cla{[1]}$ becomes

\def\aai{x\frac{1}{2} T'^2}
\def\aa{\delta(\aai)}
\begin{align}
  \cla{[1] = \i{\aa}} - \bdr \label{eq:1}
\end{align}

\def\b{\ensuremath{\clb{[2]}}}
\def\c{\ensuremath{\clc{[3]}}}
\subsubsection*{Terms \b and \c}
For \b and \c:
\def\bbi{\frac{\alpha}{2}T^2}
\def\bb{\delta(\bbi)}
\def\cci{\alpha T_{\infty} T}
\def\cc{-\delta(\cci)}
\begin{align}
  \b &= \clb{\i{\alpha \delta (\frac23 T^2)} = \i{\bb}}\label{eq:2}\\
  \c &= \clc{\i{\cc}} \label{eq:3}
\end{align}

\subsubsection*{Putting them back}
Substituting \cref{eq:1,eq:2,eq:3} back to \cref{eq:base2} gives

\newcommand{\thethree}[5]{\cla{#1} #4 \clb{#2} #5 \clc{#3}}
\begin{align}
  % \cla{[1]} + \clb{[2]} + \clc{[3]}
  \thethree{[1]}{[2]}{[3]}{+}{+}
  % &= \i{\aa} + \i{\bb} + \i{\cc} \notag\\
    &= \thethree{\i{\aa}}{\i{\bb}}{\i{\cc}}{+}{+}  - \bdr \notag\\
    % &= \i{\aa + \bb \cc} \notag\\
    &= \i{\thethree{\aa}{\bb}{\cc}{+}{}}  -\bdr \notag\\
    % &=\delta\i{\aai + \bbi - \cci} \notag\\
    &= \delta \left(\i{\thethree{\aai}{\bbi}{\cci}{+}{-}} - \Bdr \right)
      \label{eq:4}
\end{align}
Now \cref{eq:4} gives us
\begin{align}
  \mbox{The variational principle problem} =
  \begin{cases}
    I[T] &:= \i{\thethree{\aai}{\bbi}{\cci}{+}{-} - \Bdr} \\
    \delta T &= 0
  \end{cases} \label{eq:var}
\end{align}

\subsubsection*{Substitute model function}
According to Galerkin's method, if we have n nodes, then we take the following
model function $T_{t}(x)$:\footnote{Here we use the \emph{Einstein notation},
  which is the convention that says ``Repeating indexes imply summation''. For
  example, $a_i b_i = \sum_i a_i b_i$ and $K_{ij}v_j = \sum_j K_{ij} v{j}$.}

\begin{align}
  T_t(x) &:= N_i T_i & i = (1 ,\ldots , n )\label{eq:testfun}
\end{align}
Where $N_i = N_i(x)$ is the shape function for node $i$, and $T_i$ is the
temperature on node $i$.

Substituting $T = T_t$ in \cref{eq:testfun} back into \cref{eq:var} gives
\def\x{\ensuremath{\alpha T_{\infty} N_i T_i}}
\newcommand{\eBdri}[1]{\ensuremath{{[xT']}_{#1}}}  %an elaborate term
\newcommand\eBdr[1]{\ensuremath{\eBdri{#1} T(#1)}} % End of boundary
\newcommand{\twoBdr}{\ensuremath{
    \left(\cld{\eBdr{L} - \eBdr{0}}\right)
  }}
\begin{align}
  I[T_i] &:= \i{\thethree{\frac{x}{2} {(N_i T_i)'}^2}
           {\frac{\alpha}{2} {(N_i T_i)}^2}{\x}{+}{-}
           }- \twoBdr
           \notag\\
         &=\thethree{\i{T_i \frac{x}{2} N_i' N_j' T_j}}
           {\i{\frac{\alpha}{2} T_i N_i N_j T_j}}
           {\i{\x}}{+}{-} - \cld{\dotsb}
           \notag\\
         &= \thethree{T_i \i{ \frac{x}{2} N_i' N_j' } T_j}
           {T_i \i{\frac{\alpha}{2} N_i N_j} T_j}
           {\alpha T_{\infty} \i{N_i} T_i}{+}{-} - \cld{\dotsb}
           \label{eq:6}
\end{align}
Now, define
\begin{align*}
  \cla{K_{ij}} & := \cla{\i{xN_i' N_j'}}     \\
  \clb{L_{ij}} & := \clb{\i{\alpha N_i N_j}} \\
  \clc{b_i}    & := \clc{\alpha T_{\infty} \i{N_i}}  \\
  \cld{c} &:= \twoBdr{} \\
  M_{ij} & := L_{ij} + K_{ij}
\end{align*}
Then \cref{eq:6} becomes
\def\c{\cld{c}}
\begin{align*}
  I[T_i] &= \thethree{\frac{1}{2} T_i K_{ij} T_j}
           {\frac{1}{2} T_i K_{ij} T_j}{b_i T_i}{+}{-} - \c \\
         &= \frac{1}{2} T_i M_{ij} T_j - b_i T_i  - \c
\end{align*}

\def\T{\mathbf{T}}
\def\M{\mathbf{M}}
\def\b{\mathbf{b}}
\def\c{\mathbf{c}}

Setting the derivatives with respect to $\T := [T_1, \dotsc , T_{n}]$ equal to
zero gives:
\newcommand{\pFrac}[2]{\ensuremath{\frac{\partial #1}{\partial #2}}}
\newcommand{\pcTk}{\ensuremath{\cld{\pFrac{c}{T_k}}}}
\begin{align}
 \pFrac{I[T]}{T_k} &= \frac{1}{2}[T_i M_{ik} + M_{kj}T_j] - b_k - \pcTk
                                       = M_{k,j}T_j - b_k - \pcTk \notag\\
  \frac{\partial I[T]}{\partial \T }
                   &= \M\T - \b - \c= \mathbf{0} \notag\\
  \M \T &= \b + \c \label{eq:mtb}
\end{align}
Where
\newcommand{\tsps}[1]{\ensuremath{{#1}^{\text{T}}}}
\newenvironment{tbmatrix}{\begin{bmatrix}}{\end{bmatrix}^{\text{T}}}
\begin{align*}
  \M &=
       \begin{bmatrix}
         M_{11} & \cdots& M_{1n}\\
         \vdots& \ddots &\\
         M_{n1}&& M_{nn} \\
       \end{bmatrix} &
                       \b &=
                            % \tsps{\begin{bmatrix}
                            %   b_1 & \dotsc & b_n
                            % \end{bmatrix}} \\
                            \begin{tbmatrix}
                              b_1 & \dotsc & b_n
                            \end{tbmatrix} \\
  \c &=\begin{tbmatrix}
    -\eBdri{0} & 0 & \dotsc & 0 & \eBdri{L}
       \end{tbmatrix}
\end{align*}

\subsubsection*{Calculate $\c$}
Since we use a three-node mesh as shown on \cref{fig:shpfuncs}. $\c$ is
\begin{align}
  \c &=\begin{tbmatrix}
    -\eBdri{0}&0&0&0& \eBdri{L}
  \end{tbmatrix}
                      \notag\\
  \intertext{Applying the boundary condition \cref{eq:bc1} gives}
  \c &=\begin{tbmatrix}
    0&0&0&0& \eBdri{L}
  \end{tbmatrix} \label{eq:bigc}
\end{align}
And the term $\eBdri{L}$ is another unknown that the question asked for.
\begin{figure}
  \centering
  \begin{tikzpicture}[x=1cm,y=1cm]
  \begin{scope}
    % \draw[style=help lines,step=1cm] (0,-1) grid (11,5);

    \foreach \y in {0,1,2,3,4}{%
      \foreach \x in {0,7}{%
        \draw[-latex,thick] (\x,\y) -- +(4.3,0);
        \foreach \z in {0,1,2,3,4}{
          \fill[black] (\x + \z, \y) circle (1pt);
        }
      }



      \newcommand{\myh}{0.5}
      \draw (0,4) -- ++(0,\myh) -- ++(1,-\myh);
      \draw (3,0) -- ++(1,\myh) -- ++(0,-\myh);
      \foreach \i in {0,1,2}{
        \draw (\i, 3 - \i) -- ++(1,\myh) -- ++(1,-\myh);
      }
    }

    \foreach \i in {0,1,2,3}{
      \foreach \j in {0,0.7}{
        \draw (7 + \i, 3 - \i + \j) rectangle +(1,0.3);
      }
    }
    \foreach \y in {1,2,3,4,5}{
      % Label each diagram
      \node at (5, 5 - \y) {$N_{\y}(x)$};
      \node at (5 + 7, 5 - \y) {$N_{\y}'(x)$};
    }

  \end{scope}

  \newdimen\h
  \h=3.5pt

  % Draw the two dimensions
  \begin{scope}[draw=black!50]
    \foreach \i / \j / \l in {0.5 / 4 / L, 1 / 1 / l=\frac{L}{4}}{
      \draw[thin,<->] (0,-\i) to node[anchor=north] {$\l$} +(\j,0);
      \draw (0,-\i) -- +(0,\h) -- +(0, -\h)
      (\j,-\i) -- +(0,\h) -- +(0, -\h);
    }

    \h=2pt
    \newcommand{\mydimy}[4]{%
      \draw[thin,<->] (#1,#2) to node[left] {#4} +(0,#3);
      \draw (#1,#2) -- +(-\h,0) -- +(\h,0)
      (#1,#2 + #3) --  +(-\h,0) -- +(\h,0);
    }
    % \draw[thin,<->] (-0.3,4) to node[left] {1} +(0,0.5);
    % \draw (-0.3,4) -- +(-\h,0) -- +(\h,0)
    % (-0.3,4 + 0.5) --  +(-\h,0) -- +(\h,0);
    \mydimy{-0.3}{4}{0.5}{1}
    \mydimy{7-0.3}{3}{0.3}{$\frac{1}{l}$}
    \mydimy{7-0.3}{3.7}{0.3}{$\frac{1}{l}$}
  \end{scope}

  \begin{scope}[xshift=7cm,yshift=5cm]
    \draw[-latex,thick] (0,0) -- +(4.3,0);
    \draw[thick] (0,0) to node[sloped,above]{$f(x)=x$} (4,4);
    \foreach \x in {1,2,3,4}{
      \draw (\x,0) -- (\x,\x);
    }

    \newcommand{\mynode}[3][(0,0)]{
      \node[shape=circle, fill=#3,
      fill opacity=0.5,text opacity=1,font=\footnotesize, shift={#1}] {#2};
      % \node[shape=circle, fill=gray,
      % fill opacity=0.5,text opacity=1,font=\footnotesize, shift={(1cm,1cm)}] {2};
    }
    % \foreach \i/\j/\c in {0.5/1/\mycola,1.3/2/\mycolb,2.3/3/\mycolc,3.3/4/gray}{
    %   \mynode[(\i + 0.2, \i -0.2)]{\j}{\c}
    % }
    \mynode[(0.7, 0.3)]{1}{\mycola}
    \mynode[(1.5,0.75)]{2}{\mycolb}
    \mynode[(2.5,1.25)]{3}{\mycolc}
    \mynode[(3.5,1.75)]{4}{gray}
    % \mynode[(0.5 + 0.2, 0.5 -0.2)]{2}{\mycolb}
    % \mynode[(0.5 + 0.2, 0.5 -0.2)]{3}{\mycolc}
  \end{scope}

\end{tikzpicture}
  \caption{Shape functions and their derivative}\label{fig:shpfuncs}
\end{figure}
\subsubsection*{Calculate $K_{ij}$}
The shape functions and their derivatives are presented on \cref{fig:shpfuncs}.
We first calculate the diagonal entries:


\newcommand{\mynode}[3][]{
  \begin{tikzpicture}
    \def\h{17pt}
    \useasboundingbox (0,0) rectangle (\h,\h);
    \node[shape=circle, fill=#3,
    fill opacity=0.5,text opacity=1,font=\footnotesize, shift={(8.5pt,3pt)}] {#2};
  \end{tikzpicture}
}

\newcommand{\na}{\mynode{1}{\mycola}}
\newcommand{\nb}{\mynode{2}{\mycolb}}
\newcommand{\nc}{\mynode{3}{\mycolc}}
\newcommand{\nd}{\mynode{4}{gray}}

\begin{align}
  K_{11} &= \i{x N_0' N_0'}
           = [\text{area of \na on \cref{fig:shpfuncs}}]
           \times {[\text{the slope of shape function}]}^2
           :=  \na \times s \label{eq:k11}\\
  K_{22} &= \i{xN_1' N_1'} = [\text{area of \na{} $+$ \nb{}}]
           \times {[\text{the slope of shape function}]}^2
           := (\na{} + \nb{}) \times s \label{eq:k22}\\
  \intertext{Similarly}
  K_{33} & = (\nb{} + \nc{}) \times s \label{eq:k33}\\
  K_{44} & = (\nc{} + \nd{}) \times s \label{eq:k44}\\
  K_{55} &= (\nd{}) \times s \label{eq:k55}
\end{align}
Note that in \cref{eq:k11,eq:k22,eq:k33,eq:k44,eq:k55}, they are all multiplied by the same value $s = \frac{1}{l^2}$, even if
the slope is sometimes positive and sometimes negative. This is because, when
they are squared, they are always positive.

Now the areas can be calculated as follows:
\begin{align}
  \na &= \frac{l^2}{2} \label{eq:na}\\
  \nb &= l^2 + \na = \frac{3}{2}l^2 \label{eq:nb}\\
  \nc &= l^2 + \nb = \frac{5}{2}l^2 \label{eq:nc}\\
  \nd &= l^2 + \nc = \frac{7}{2}l^2 \label{eq:nd}
\end{align}
Substituting \cref{eq:na,eq:nb,eq:nc,eq:nd} back into
\cref{eq:k11,eq:k22,eq:k33,eq:k44,eq:k55} gives
\begin{align}
  K_{11} &=  \na \times s =
           \frac{l^2}{2} \times \frac{1}{l^2}
           = \frac{1}{2}\label{eq:nk11}\\
  K_{22} &= (\na{} + \nb{}) \times s =
           (\frac{l^2}{2} + \frac{3}{2}l^2) \frac{1}{l^2} =
           \frac{1}{2} + \frac{3}{2} = \frac{4}{2}
           \label{eq:nk22}\\
  K_{33} & = (\nb{} + \nc{}) \times s
           (\frac{3}{2}l^2 + \frac{5}{2}l^2) \frac{1}{l^2} =
           \frac{3}{2} + \frac{5}{2} = \frac{8}{2}
           \label{eq:nk33}\\
  K_{44} & = (\nc{} + \nd{}) \times s =
           \frac{5}{2} + \frac{7}{2} = \frac{23}{2}
           \label{eq:nk44}\\
  K_{55} &= \nd{} \times s = \frac{7}{2}
           \label{eq:nk55}
\end{align}

Now for the off-diagonal entries, because
\def\K{\ensuremath{\mathbf{K}}}
\begin{enumerate}
\item $\K$ is symmetrical ($K_{ij} = K_{ji}$)
\item the product of two node's shape functions is zero if the nodes are not
  next to each other
\end{enumerate}
the only entries that we need to calculate are the following
\begin{align}
  K_{12} &
           = [\text{area of \na{} on \cref{fig:shpfuncs}}]
           \times - {[\text{the slope of shape function}]}^2
           = \na{} \times (-s)  = -\frac{1}{2}\label{eq:nk12}\\
  K_{23} &= \nb{} \times (-s) = -\frac{3}{2}\label{eq:nk23}\\
  K_{34} &= \nc{} \times (-s) = -\frac{5}{2}\label{eq:nk34}\\
  K_{45} &= \nd{} \times (-s) = -\frac{7}{2}\label{eq:nk45}
\end{align}

Combining
\cref{eq:nk12,eq:nk23,eq:nk34,eq:nk45,eq:nk11,eq:nk22,eq:nk33,eq:nk44,eq:nk55}
gives the matrix $\K$
\def\matK{\ensuremath{\frac{1}{2}
    \begin{bmatrix}
      1 & -1 &&&\\
      -1 &4&-3&&\\
      & -3 &8&-5&\\
      &&-5&12&-7\\
      &&&-7&7
    \end{bmatrix}
  }}
\begin{align}
  \K &=
               \begin{bmatrix}
                 K_{11} & K_{12} &&&\\
                 K_{12} & K_{22} &K_{23}&&\\
                 & K_{23} &K_{33}&K_{34}&\\
                 &&K_{34}&K_{44}&K_{45}\\
                 &&&K_{45}&K_{55}
               \end{bmatrix} = \matK{} \label{eq:bigK}
\end{align}
In \cref{eq:bigK}, empty entry implies zero.

\subsubsection*{Calculate $L_{ij}$}
\def\L{\ensuremath{\mathbf{L}}}
Before we calculate \L, we calculate two \emph{convolution integrals}
% Triangle A
\newcommand{\Ta}{
  \begin{tikzpicture}
    \tikzstyle{every node}=[font=\scriptsize]
    \def\w{1cm}
    \def\h{0.5cm}
    \useasboundingbox (0,0) rectangle (\w + 15pt, \h);
    \fill[gray] (0,0) -- (\w,\h) -- (\w,0) -- cycle;
    % Draw the two dimensions
    % \def\h{3.5pt}
    \def\i{0.1cm}
    \draw[thin,<->] (0,-\i) to node[below] {$l$} +(\w,0);
    \draw[thin,<->] (\w + \i,0) to node[right] {1} +(0,\h);
  \end{tikzpicture}
}
% Triangle B
\newcommand{\Tb}{
  \begin{tikzpicture}
    \tikzstyle{every node}=[font=\scriptsize]
    \def\w{1cm}
    \def\h{0.5cm}
    \useasboundingbox (-10pt,0) rectangle (\w + 5pt, \h);
    \fill[gray] (0,0) -- (0,\h) -- (\w,0) -- cycle;
    \def\i{0.1cm}
    \draw[thin,<->] (0,-\i) to node[below] {$l$} +(\w,0);
    \draw[thin,<->] (-\i,0) to node[left] {1} +(0,\h);
  \end{tikzpicture}
}

\newcommand{\il}[1]{\ensuremath{\int_0^l #1 dx}}
\newcommand{\brkIt}[1]{\ensuremath{\left[#1\right]}}
\newcommand{\brkItl}[1]{\ensuremath{{\brkIt{#1}}_0^l}}
\newcommand{\prnIt}[1]{\ensuremath{\left( #1 \right)}}
\def\xl{\ensuremath{{\left( \frac{x}{l} \right)}^2}}
% \def\a{\cla{a}}
% \def\b{\clb{b}}
\def\a{\tcbox[on line,size=small,colback=\mycola!20,colframe=\mycola]{a}}
\def\b{\tcbox[on line,size=small,colback=\mycolb!20,colframe=\mycolb]{b}}
\def\anum{\ensuremath{\alpha\frac{l}{3}}}
\def\bnum{\ensuremath{\alpha\frac{l^2}{2} - \anum }}
\def\bnumb{\ensuremath{\alpha\frac{l}{6}(3l-2)}}
\begin{align}
  \a &:= \alpha \il{\Ta \Ta} = \alpha  \il{\xl{}} \notag\\
  &= \alpha \frac{1}{l^2} \brkItl{\frac{1}{3}x^3}
  = \frac{l^3}{3l^2} = \anum \\[\baselineskip]
  \b &:= \alpha \il{\Ta \Tb } = \alpha  \il{\frac{x}{l} (l - \frac{x}{l})} \notag\\
    &= \alpha\il{x} - \alpha \il{\xl{}}
      = \alpha\brkItl{\frac{x^2}{2}} - \a \notag\\
  &= \bnum =  \bnumb
\end{align}
Now, because $L_{ij} := \alpha \i{N_i N_j}$, from \cref{fig:shpfuncs} we see
that
\begin{align}
  L_{11} &= L_{55} = \a  = \anum \label{eq:l1}\\
  L_{22} &= L_{33} = L_{44} = 2 \times \a = \alpha\frac{2l}{3} \label{eq:l2}\\
  L_{12} &= L_{23} = L_{34} = L_{45} = \b = \bnumb \label{eq:l3}
\end{align}
\def\L{\ensuremath{\mathbf{L}}}
Since \L \ has the same properties as \K \
(i.e.,being symmetrical, non-neibour entries have value of 0).
Combining \cref{eq:l1,eq:l2,eq:l3} gives the matrix \L
\def\x{3l-2}
\begin{align}
  \mathbf{L} &=
               \begin{bmatrix}
                 L_{11} & L_{12} &&&\\
                 L_{12} & L_{22} &L_{23}&&\\
                 & L_{23} &L_{33}&L_{34}&\\
                 &&L_{34}&L_{44}&L_{45}\\
                 &&&L_{45}&L_{55}
               \end{bmatrix}
                            =
                            \frac{\alpha l}{6}
                            \begin{bmatrix}
                              2& c &&&\\
                              c &4&c&&\\
                              & c &4&c&\\
                              &&c&4&c\\
                              &&&c&2
                            \end{bmatrix} \label{eq:bigL}\\
  c &:= \x \notag
\end{align}
In \cref{eq:bigL}, empty entry implies zero.

\subsubsection*{Calculate $b_i$}
\def\b{\ensuremath{\mathbf{b}}}
\def\matb{\ensuremath{
    \begin{bmatrix}
      1 \\ 2  \\ 2 \\ 2 \\ 1
    \end{bmatrix}
  }}
The last array needed for \cref{eq:mtb} is \b \ where
\def\aTi{\ensuremath{\alpha T_{\infty}}}
\def\A{\tcbox[on line,size=small,colback=\mycolc!20,colframe=\mycolc]{A}}
\def\alT{\ensuremath{\alpha T_{\infty}}}
\begin{align}
  \clc{b_i} :&= \clc{\aTi \i{N_i}} \notag\\
             &= \aTi  \times [\text{The area under shape function \ } N_i(x)] \notag\\
  \intertext{Therefore}
  \b &= \A \matb
       \quad \text{where \ } \A = \alT \times \left[
       \text{area of \Ta}  = \frac{l}{2} \right] \label{eq:bigb}
\end{align}

\subsubsection*{Back to the system}
Now, substituting \cref{eq:bigc,eq:bigK,eq:bigL,eq:bigb} back into \cref{eq:mtb} gives
the system
\def\M{\ensuremath{\mathbf{M}}}
\def\matM{\ensuremath{
    \begin{bmatrix}
      \frac{\alpha}{3} + \frac{1}{2} & \frac{\alpha}{6} - \frac{1}{2} &&&\\
      \frac{\alpha}{6} - \frac{1}{2} & \frac{2 \alpha}{3} + 2& \frac{\alpha}{6} - \frac{3}{2} &&\\
       &\frac{\alpha}{6} - \frac{3}{2}&\frac{2 \alpha}{3} + 4& \frac{\alpha}{6} - \frac{5}{2} &\\
       && \frac{\alpha}{6} - \frac{5}{2}&\frac{2 \alpha}{3} + 6& \frac{\alpha}{6} - \frac{7}{2}\\
       &&& \frac{\alpha}{6} - \frac{7}{2}&\frac{\alpha}{3} + \frac{7}{2}
      \end{bmatrix}
  }}
\def\matMmoved{\ensuremath{
    \begin{bmatrix}
      \frac{\alpha}{3} + \frac{1}{2} & \frac{\alpha}{6} - \frac{1}{2} &&\\
      \frac{\alpha}{6} - \frac{1}{2} & \frac{2 \alpha}{3} + 2& \frac{\alpha}{6} - \frac{3}{2} &\\
      &\frac{\alpha}{6} - \frac{3}{2}&\frac{2 \alpha}{3} + 4& \frac{\alpha}{6} - \frac{5}{2} \\
      && \frac{\alpha}{6} - \frac{5}{2}&\frac{2 \alpha}{3} + 6\\
      &&& \frac{\alpha}{6} - \frac{7}{2}
    \end{bmatrix}
  }
}

\begin{align}
  \M &= \L + \K =
  \frac{\alpha l}{6}
  \begin{bmatrix}
    2& c &&&\\
    c &4&c&&\\
    & c &4&c&\\
    &&c&4&c\\
    &&&c&2
  \end{bmatrix} + \matK \notag\\
  c &= \x \notag\\
  \b &=\alT \frac{l}{2}\begin{bmatrix}1&2&2&2&1\end{bmatrix}^{\text{T}} \notag\\
  \intertext{Substituting $l = \frac{4}{4} = 1$
  gives}
  c &= 1 \notag\\
  \M &=
       \frac{\alpha}{6}
       \begin{bmatrix}
         2& 1 &&&\\
         1 &4&1&&\\
         & 1 &4&1&\\
         &&1&4&1\\
         &&&1&2
       \end{bmatrix} + \matK \notag\\
     &=\MyGet{M} \label{eq:bigM}\\
  \b &=\alT \frac{1}{2}\begin{bmatrix}1&2&2&2&1\end{bmatrix}^{\text{T}} \notag \\
\end{align}
Therefore the system in \cref{eq:mtb} becomes
\begin{align}
  \MyGet{M}
  \begin{bmatrix} T_1 \\ T_2 \\ T_3 \\ T_4 \\ T_5 \end{bmatrix}
  &= \alT \frac{1}{2} \matb +
    \begin{bmatrix}
      0\\0\\0\\0\\ s
    \end{bmatrix} \label{eq:readyToSplit}\\
  s&= \eBdri{L} \notag
\end{align}
Because of boundary condition \cref{eq:bc2}, $T_5=T_0$ is known, therefore the
unknowns are:
\begin{itemize}
\item the node temperatures $T_1$ to $T_4$. [4 unknowns]
\item $s$,the heat flow per unit time at $x=L$ [1 unknown]
\end{itemize}
Therefore there are 5 equations and 5 unknowns.
We follow the following steps:
\begin{enumerate}
\item Solve for $T_1$ to $T_4$
\item Find $s$, 
\end{enumerate}
\subsubsection*{Solve for $T_1$ to $T_4$}
Moving $T_5$ to the RHS of \cref{eq:readyToSplit} gives
\def\Tn{\ensuremath{\begin{bmatrix} T_1 \\ T_2 \\ T_3 \\ T_4 \end{bmatrix}}}
\def\alTt{\ensuremath{\alT \frac{1}{2}}}
\def\g#1{\MyGet{#1}}
\begin{align}
  \g{M2} \Tn
    &= \alTt \matb +
    \begin{bmatrix}
      0\\0\\0\\0\\ s
    \end{bmatrix} -
  T_5\g{Me} \label{eq:toDivide}\\
  s&= \eBdri{L} \notag
\end{align}
We divide \cref{eq:toDivide} into two pieces for our two steps:
\begin{align}
  \g{M.upper} \Tn &= \alTt \g{v.top} - T_5 \g{v2.top} \label{eq:mupper}\\
  \prnIt{\g{last.lhs}} T_4 &= \alTt + s  - T_5 \prnIt{\g{last.rhs}} \label{eq:mlower}
\end{align}
Substituting $\alpha = 0.4168, T_{\infty} = 75$ and $T_5 = T_0 = 250$ into
\cref{eq:mupper} gives
\begin{align}
  \g{M.upper.n} \Tn &= \g{rhs.upper} = \g{rhs.upper.n} \notag\\
  \intertext{Solving the system gives}
  \Tn &= \g{T1.T4} \label{eq:resultT}
\end{align}
Now substituting $T_4 = \g{T4.n}$ into \cref{eq:mlower} gives
\begin{align}
  s &= \eBdri{L} \notag\\
    &= \prnIt{\g{last.lhs}} \times T_4 + T_5 \prnIt{\g{last.rhs}} - \alTt \notag\\
    &= \prnIt{\g{last.lhs.n}} \times \g{T4.n} + 250 \times \g{last.rhs.n} - \g{alT2.n} \notag\\
  &= \g{s}
\end{align}
Which is the \emph{heat flow per unit time at $x=L$}
\subsubsection*{Remark}
As shown in \cref{eq:resultT}, the temperature $T$ gradually increases from $T_1 = \g{T1.n}$,
which is a bit higher than the ambient temperature $T_{\infty}$, to $T_4 =
\g{T4.n}$, which is a bit lower than the temperature, $T_5 = T_0 = 250$, at the
end $x=L$.

In order to be more accurate, finer mesh (i.e., more nodes) shall be used.
