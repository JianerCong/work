\subsection{Flexural Design of the Beam}
By assuming that the depth of neutral axis $x_{\mbox{assumed}} = 0.25d = 0.25
\times \g{d} = \g{xa}$, for the required steel area $A_{s,required}$, we have
\footnote{We have assumed that compression steel and tensile steel have the same area.}:
\begin{align*}
  A_{s,required} &= \frac{M_{Ed}}{f_{yd}0.9d}
                   =  \frac{\g{MEd}}{\g{fyd} \times 0.9  \times \g{d}}
                   = \g{Asr} \mbox{mm}^{2}
\end{align*}

Therefore, we can use \g{n14} 14mm-diamter bars and \g{n18}
18mm-diameter bars, which gives an $A_{s} = \g{As}$ mm $^{2}$. Solving the
following quadratic equation we can get the neutral axis $x$:

\begin{align*}
  a x^{2} + b x + c &= 0 \\
  \\[-1cm]               %remove the extra spacing
  \intertext{Where}
  \\[-1cm]               %remove the extra spacing
  a &= 0.8Bf_{cd}
      = 0.8 \times \g{b}\mbox{mm} \times \g{fcd} \mbox{Nmm}^{-2} \\
                    &= \g{qea} \mbox{Nmm}^{-1}\\
  b &= A_{s} (E_{s} \epsilon_{cu} - f_{yd})
      = \g{As}\mbox{mm}^{2} (\g{Es}\mbox{Nmm}^{-2}
      \times
      \g{epcu} - \g{fyd}\mbox{Nmm}^{-2}) \\
                    &= \g{qeb} \mbox{N}\\
  c &= - A_{s} E_{s} \epsilon_{cu} c
      = \g{As} \mbox{mm}^{2} \times \g{Es} \mbox{Nmm}^{-2}
      \times \g{epcu} \times \g{c} \mbox{mm} \\
                    &= \g{qec} \mbox{Nmm}\\
  \\[-1cm]               %remove the extra spacing
  \intertext{Therefore}
  \\[-1cm]               %remove the extra spacing
  x &= \g{x} \mbox{mm}
\end{align*}

Checking the compressive and tensile steel strain assumption:
\begin{align*}
  \epsilon_{s} &= \frac{\epsilon_{cu}}{x} (d - x) 
               =  \frac{\g{epcu}}{\g{x}\mbox{mm}} 
                 \times (\g{d} \mbox{mm} - \g{x} \mbox{mm})
                 = \g{eps}
               > \epsilon_{sy} = 0.0021 \mbox{(verified)} \\
  \epsilon_{s}' &= \frac{\epsilon_{cu}}{x} \times (x - c) 
               = \frac{\g{epcu}}{\g{x}\mbox{mm}} \times
                  (\g{x}\mbox{mm} - \g{c}\mbox{mm})
                  = \g{epsp}
                  < \epsilon_{sy} = 0.0021 \mbox{(verified)}
\end{align*}

So $M_{Rd}$ is
\begin{align*}
  M_{Rd} &= A_{s} (f_{yd} (d - 0.4 x) +
           E_{s} \epsilon_{cu} \frac{x-c}{x} (0.4x - c)) \\
         &=
           \g{As} \mbox{mm}^{2}(
           \g{fyd} \mbox{Nmm}^{-2}
           (\g{d} \mbox{mm} - 0.4 \times \g{x}\mbox{mm}) \\
           &+
           (
           \g{Es} \mbox{Nmm}^{-2} \times \g{epcu}
           \frac{\g{x} \mbox{mm}- \g{c} \mbox{mm}}
           {\g{x} \mbox{mm}}
           (0.4 \times \g{x} \mbox{mm} - \g{c} \mbox{mm})
           )) \\
  &= \g{MRd} \mbox{Nmm}
\end{align*}
So we have
\begin{align*}
  M_{Rd} &= \g{MRd} > M_{Ed}^{+} = \g{MEd} \mbox{Nmm}\\
  M_{Rd} &= \g{MRd} > M_{Ed}^{-} = \g{MEdl} \mbox{Nmm}
\end{align*}
\subsection{Beam Local Ductility Check}
\subsubsection{Ratio of compressive reinforcement}
The ratio of compressive reinforcement $\frac{A_{s}'}{A_{s}}$ should be greater
than 0.5. However, this can be skipped since \(A_{s} = A_{s}'\).

\subsubsection{Check $\rho < \rho_{\max}$}
Next is to verify that
\begin{align*}
  \rho &< \rho_{\max} := \rho' +
         \frac{0.0018f_{cd}}{\mu_{\phi}\epsilon_{sy}f_{yd}}\\
  \\[-1cm]               %remove the extra spacing
  \intertext{Where}
  \\[-1cm]               %remove the extra spacing
  \mu_{\phi} &=
               \begin{cases}
                 2q_{0} - 1 & \text{if } T_1 \geq T_{C} \\
                 1 + 2(q_{0} - 1)\frac{T_c}{T_1} & \text{if } T_{1} < T_{C}
               \end{cases}
\end{align*}
However, this check can also be skipped, because $A_s = A_s'$ so $\rho = \rho'$,
and $\rho_{\max}$ is $\rho'$ plus something, namely
$\frac{0.0018f_{cd}}{\mu_{\phi}\epsilon_{sy}f_{yd}}$, let's denote it by $k$.
However, we observe that according to \S EC-8 5.2.2.2, for \emph{multistorey,
  multibay frame building} with DCH, $q_0 = 4.5 \times \frac{\alpha_u}{\alpha_1}
= 4.5 \times 1.3 = 5.85$. This implies that both $2q_0-1$ and
$1+2(q_0-1)\frac{T_c}{T_1}$ are positive, which implies that $\mu_{\phi}$ is
always positive no matter whether $T_1 \geq T_c$ or $T_1 < T_c$, so $k$ is
always positive, so $\rho_{\max} = \rho + k$ is always greater than $\rho$. (i.e
This check can be skipped.)

\subsubsection{Check $\rho > \rho_{\min}$}
Now we calculate $\rho = \frac{A_s}{bd} = \frac{
  \g{As}}{\g{b} \times \g{d}} = \g{rho}$ and $\rho_{\min}$ can
be calculated as:
\begin{align*}
  \rho_{\min} &= 0.5 \frac{f_{ctm}}{f_{yk}} \\
  \\[-1cm]               %remove the extra spacing
  \intertext{Where}
  \\[-1cm]               %remove the extra spacing 
  f_{ctm} &= 0.3 \times f_{ck}^{\frac{2}{3}} \\
  \\[-1cm]               %remove the extra spacing
  \intertext{Therefore}
  \\[-1cm]               %remove the extra spacing
  \rho_{\min} &= 0.5 \frac{0.3 \times 25 ^{\frac{2}{3}}}{500} = 0.0026 < \rho
                \mbox{(verified)}
\end{align*}
\subsection{Shear Strength and Confinement}
According to \S EC8 5.4.2.2, beams demand shear forces should always be
determined as per the \emph{capacity design rule}, that includes both:
\begin{itemize}
\item The shear resulting from the gravity loads ($V_{Ed,g}$) that are within
  seismic load combinations $G_{ki} + \Phi_i Q_{ki}$.
\item The shear produced by the end moments $M_{i,d}$ ($i = 1,2$ denotes the
  beam end section) corresponding plastic hinge formation ($V_{Ed,E}$).
\begin{align*}
  M_{id} &= \gamma_{Rd} M_{Rd,c,i} \min (1,\frac{\sum M_{Rd,col}}{\sum M_{Rd,beam}}) \\
         &= \gamma_{Rd} M_{Rd,c,i} 
\end{align*}
\item The shear demand distribution as per the previous rule is presented in
  Figure~\ref{fig:shrd}.
\end{itemize}

\begin{figure}
  \centering
  \includegraphics[width=0.7\textwidth]{shrd}
  \caption{Shear demand distribution (Courtesy: Karim Al-Jawhari)}\label{fig:shrd}
\end{figure}

\newcommand{\Va}{\frac{G_k+\psi_EQ_k l_{clear}}{2}}
\newcommand{\Vb}{\frac{\gamma_{Rd}(M_{L,Rd}^+ + M_{R,Rd}^- )}{l_{clear}}}
\newcommand{\Vc}{\frac{\gamma_{Rd}(M_{L,Rd}^- + M_{R,Rd}^+ )}{l_{clear}}}
The formulas required to calculate the final shear demands are
\begin{align*}
  V_{left, \rightarrow } &= \Va - \Vb \\
  V_{right, \rightarrow } &= -\Va - \Vb \\
  V_{left, \leftarrow } &= \Va + \Vc \\
  V_{right, \leftarrow } &= -\Va + \Vc \\
\end{align*}
\subsubsection{Critical region}
\begin{enumerate}
\item Set $V_{Rd,s} = V_{Ed} = \g{VEd}$ kN.
\item Assuming using a 2-leg hoop with size \g{dh} mm. For the critical
  region, we assume that $\theta = 45 ^{\circ}$ and $\alpha = 90 ^{\circ}$ (i.e.
  the traverse reinforcement (hoops) is vertical).
  \begin{align*}
    s_{required} &= 0.9d \frac{A_{sw}}{V_{Ed}} f_{yd}
                   (\cot \theta + \cot \alpha)\sin \alpha \\
                 &= 0.9d \frac{A_{sw}}{V_{Ed}} f_{yd} (1 + 0) \times 1\\
                 &= 0.9 \times \g{d} \mbox{mm}
                   \times
                   \frac{\g{Asw}\mbox{mm}^{2}}{\g{VEd} \mbox{kN}}
                   \times \g{fyd} \mbox{Nmm}^{-2} \\
                 &= \g{sreq} \mbox{mm}
  \end{align*}
\item According to \S EC-8 5.4.3.1.2 (6)(b) $s$ should not exceed
  \begin{align*}
    s_{\max} &= \min \{ \frac{h_{beam}}{4}, 24\phi_{sw},
               225\mbox{mm}, 8\phi_{long} \} \\
             &= \min \{ \frac{500}{4} \mbox{mm}, 24 \times \g{dh} \mbox{mm},
               225\mbox{mm}, 8 \times 14 \mbox{mm}\} \\
             &= \g{smax}
               \mbox{mm} > s_{required} = \g{sreq} \mbox{mm (Verified)} \\
  \end{align*}
\item To make the spacing practical, we set $s_{actual} = \g{sact}$ mm, and
  the corresponding $V_{Rd, s, actual} $ is
\begin{align*}
  V_{Rd, s, actual}&= 0.9d \frac{A_{sw}}{s_{actual}}f_{yd}\cot  \theta \\
  &= 0.9 \times \g{d} \mbox{mm}
  \times \frac{\g{Asw} \mbox{mm}^{2}}{\g{sact} \mbox{mm}}
  \times \g{fyd} \mbox{Nmm}^{-2} \times 1 \\
  &= \g{VRdsAct} \mbox{kN}
\end{align*}
  which is greater than $V_{Ed} = \g{VEd} $kN (verified).
\end{enumerate}

\subsubsection{Non-Critical Region}
For the non-critical region we assume the following:
\begin{itemize}
\item The struct angle $\theta = 21.8^{\circ}$.
\item The design shear force is same as the critical region: $V_{Ed} =
  \g{VEd}$ kN.
\end{itemize}
It is also assumed that the hoops are the same as the critical region, therefore
we have
\begin{align*}
                 s_{required} &= 0.9d \frac{A_{sw}}{V_{Ed}} f_{yd}
                                (\cot \theta + \cot \alpha)\sin \alpha \\
                              &= 0.9d \frac{A_{sw}}{V_{Ed}} f_{yd} (\cot 21.8^{\circ} +
                                \cot 90 ^{\circ}) \times \sin 90^{\circ}\\
               &= 0.9 \times \g{d} \mbox{mm}
                 \times
                 \frac{\g{Asw}\mbox{mm}^{2}}{\g{VEd} \mbox{kN}}
                 \times \g{fyd} \mbox{Nmm}^{-2}  \times (2.5 + 0) \times 1\\
               &= \g{sreqn} \mbox{mm}
\end{align*}

When considering the non-critical regions, the requirements to be followed are
from EC2 rather than EC8 in terms of spacing and minimum required amount.

According to \S EC2 9.2.2:
\begin{enumerate}
\item $s$ should be smaller than $s_{\max} := \frac{0.75d}{1+\cot\alpha} =
  0.75 \times \g{d} = \g{smaxn} > s_{required}$ (satisfied)
\item the ratio of shear reinforcement ($\rho_w = \frac{A_{sw}}{sb \sin
    \alpha}$) in the non-critical regions should be greater than
\begin{align*}
  \rho_{w,\min} &= \frac{0.08 \sqrt{f_{ck}}}{f_{yk}} = 0.00084 \\
  \rho_w &= \g{rho} > \rho_{w,\min} = 0.00084 \mbox{(satisfied)}
\end{align*}
\item The spacing \g{sreqn} mm is not pratical. For a practical spacing, we
  set $s_{actual} = \g{sactn}$ mm, and the corresponding $V_{Rd, s, actual}$
  is
\begin{align*}
  V_{Rd, s, actual} &= 0.9d \frac{A_{sw}}{s_{actual}}f_{yd}\cot  21.8 ^{\circ} \\
                    &= 0.9 \times \g{d} \mbox{mm} \times \frac{\g{Asw}
                      \mbox{mm}^{2}}{\g{sactn} \mbox{mm}}
                      \times \g{fyd} \mbox{Nmm}^{-2} \times 2.5 \\
                    &= \g{VRdsActn} \mbox{kN}
\end{align*}
  which is greater than $V_{Ed} = \g{VEd}$kN (verified).
\end{enumerate}