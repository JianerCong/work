\documentclass[fleqn,10pt]{olplainarticle}
% Use option lineno for line numbers 

\title{RC Member Design Report}
\graphicspath{ {./fig/} }
\author[1]{Jianer Cong}
\usepackage{tikz}
% \usepackage{placeins}
\usepackage{afterpage}

\def\MyPattern#1{My Random Prefix and #1}
\def\MySet#1#2{\expandafter\def%
  \csname\MyPattern{#1}\endcsname%
  {#2}
}
\def\MyGet#1{\csname\MyPattern{#1}\endcsname}
% \MySet{MEdl}{\textcolor{black}{45780000}}
% \input{data/datain}
\MySet{t1.L}{\textcolor{blue}{5}}
\MySet{t1.b}{\textcolor{blue}{0.5}}
\MySet{t1.h}{\textcolor{blue}{1.75}}
\MySet{t1.P}{\textcolor{blue}{100}}
\MySet{t1.k}{\textcolor{blue}{0.833}}
\MySet{t1.I}{\textcolor{blue}{0.2233}}
\MySet{t1.Vy}{\textcolor{blue}{0.0006451}}
\MySet{t1.A}{\textcolor{blue}{0.875}}
\MySet{t1.E}{\textcolor{blue}{3.148 \ensuremath{\times 10^{7}}}}
\MySet{t1.G}{\textcolor{blue}{1.311 \ensuremath{\times 10^{7}}}}
\MySet{t1.Vy.mm}{\textcolor{blue}{0.6451}}
\MySet{t2.E}{\textcolor{blue}{3.148 \ensuremath{\times 10^{0}}}}
\MySet{t2.I}{\textcolor{blue}{0.06337}}
\MySet{t2.L}{\textcolor{blue}{12}}
\MySet{t2.b}{\textcolor{blue}{0.5}}
\MySet{t2.h}{\textcolor{blue}{1.15}}
\MySet{t2.m}{\textcolor{blue}{1380}}
\MySet{t2.fn}{\textcolor{blue}{4.677}}
\MySet{t2.f1}{\textcolor{blue}{0.5602}}
\MySet{t2.f2}{\textcolor{blue}{69.7}}
\MySet{t3.L}{\textcolor{blue}{48}}
\MySet{t3.b}{\textcolor{blue}{1}}
\MySet{t3.h}{\textcolor{blue}{8}}
\MySet{t3.P}{\textcolor{blue}{40}}
\MySet{t3.k}{\textcolor{blue}{0.833}}
\MySet{t3.I}{\textcolor{blue}{42.67}}
\MySet{t3.Vy}{\textcolor{blue}{1.217}}
\MySet{t3.A}{\textcolor{blue}{8}}
\MySet{t3.E}{\textcolor{blue}{2.901 \ensuremath{\times 10^{4}}}}
\MySet{t3.G}{\textcolor{blue}{1.116 \ensuremath{\times 10^{4}}}}
\MySet{t3.Vy.mm}{\textcolor{blue}{1217}}

\usepackage{booktabs}

\affil[1]{zcesjco@ucl.ac.uk}
\keywords{RC Member design, Seismic Design}

\begin{abstract}
  This report presents the design of the selected column and beams on the
  assigned frame according to EC8 and EC2. The ductility class DCH is applied.
  The material assumed are C25/30 concrete and Grade 500 steel.
\end{abstract}

\begin{document}
\maketitle

\section*{Introduction}
The building geometry is shown in Figure~\ref{fig:fr3d}. Three elements are to
be designed, the column C1 on the ground floor, beam B1 and beam B2 of frame 1
as shown in Figure~\ref{fig:frl}. The section dimensions for the ground storey
columns are $400 \mbox{mm} \times 600 \mbox{mm}$ and for the first storey beams
are  $300 \mbox{mm} \times 500 \mbox{mm}$.
\begin{itemize}
\item The design of B1 is presented in Section \ref{sec:bm}.
\item The design of B2 is presented in Section \ref{sec:bm2}.
\item The design of C1 is presented in Section \ref{sec:col}.
\end{itemize}

\begin{figure}
  \centering
  \includegraphics[width=0.7\textwidth]{frame3D}
  \caption{Overview of the RC building geometry}\label{fig:fr3d}
\end{figure}

\begin{figure}
  \centering
  \includegraphics[width=0.7\textwidth]{frame-labeled}
  \caption{Elements to be designed: beams B1,B2 column C1}\label{fig:frl}
\end{figure}

\section{Design of Beam B1}\label{sec:bm}
\newcommand{\g}[1]{\MyGet{#1}}
\input{tex/bm}

\section{Design of Beam B2}\label{sec:bm2}
\renewcommand{\g}[1]{\MyGet{b2.#1}}
\subsection{Flexural Design of the Beam}
By assuming that the depth of neutral axis $x_{\mbox{assumed}} = 0.25d = 0.25
\times \g{d} = \g{xa}$, for the required steel area $A_{s,required}$, we have
\footnote{We have assumed that compression steel and tensile steel have the same area.}:
\begin{align*}
  A_{s,required} &= \frac{M_{Ed}}{f_{yd}0.9d}
                   =  \frac{\g{MEd}}{\g{fyd} \times 0.9  \times \g{d}}
                   = \g{Asr} \mbox{mm}^{2}
\end{align*}

Therefore, we can use \g{n14} 14mm-diamter bars and \g{n18}
18mm-diameter bars, which gives an $A_{s} = \g{As}$ mm $^{2}$. Solving the
following quadratic equation we can get the neutral axis $x$:

\begin{align*}
  a x^{2} + b x + c &= 0 \\
  \\[-1cm]               %remove the extra spacing
  \intertext{Where}
  \\[-1cm]               %remove the extra spacing
  a &= 0.8Bf_{cd}
      = 0.8 \times \g{b}\mbox{mm} \times \g{fcd} \mbox{Nmm}^{-2} \\
                    &= \g{qea} \mbox{Nmm}^{-1}\\
  b &= A_{s} (E_{s} \epsilon_{cu} - f_{yd})
      = \g{As}\mbox{mm}^{2} (\g{Es}\mbox{Nmm}^{-2}
      \times
      \g{epcu} - \g{fyd}\mbox{Nmm}^{-2}) \\
                    &= \g{qeb} \mbox{N}\\
  c &= - A_{s} E_{s} \epsilon_{cu} c
      = \g{As} \mbox{mm}^{2} \times \g{Es} \mbox{Nmm}^{-2}
      \times \g{epcu} \times \g{c} \mbox{mm} \\
                    &= \g{qec} \mbox{Nmm}\\
  \\[-1cm]               %remove the extra spacing
  \intertext{Therefore}
  \\[-1cm]               %remove the extra spacing
  x &= \g{x} \mbox{mm}
\end{align*}

Checking the compressive and tensile steel strain assumption:
\begin{align*}
  \epsilon_{s} &= \frac{\epsilon_{cu}}{x} (d - x) 
               =  \frac{\g{epcu}}{\g{x}\mbox{mm}} 
                 \times (\g{d} \mbox{mm} - \g{x} \mbox{mm})
                 = \g{eps}
               > \epsilon_{sy} = 0.0021 \mbox{(verified)} \\
  \epsilon_{s}' &= \frac{\epsilon_{cu}}{x} \times (x - c) 
               = \frac{\g{epcu}}{\g{x}\mbox{mm}} \times
                  (\g{x}\mbox{mm} - \g{c}\mbox{mm})
                  = \g{epsp}
                  < \epsilon_{sy} = 0.0021 \mbox{(verified)}
\end{align*}

So $M_{Rd}$ is
\begin{align*}
  M_{Rd} &= A_{s} (f_{yd} (d - 0.4 x) +
           E_{s} \epsilon_{cu} \frac{x-c}{x} (0.4x - c)) \\
         &=
           \g{As} \mbox{mm}^{2}(
           \g{fyd} \mbox{Nmm}^{-2}
           (\g{d} \mbox{mm} - 0.4 \times \g{x}\mbox{mm}) \\
           &+
           (
           \g{Es} \mbox{Nmm}^{-2} \times \g{epcu}
           \frac{\g{x} \mbox{mm}- \g{c} \mbox{mm}}
           {\g{x} \mbox{mm}}
           (0.4 \times \g{x} \mbox{mm} - \g{c} \mbox{mm})
           )) \\
  &= \g{MRd} \mbox{Nmm}
\end{align*}
So we have
\begin{align*}
  M_{Rd} &= \g{MRd} > M_{Ed}^{+} = \g{MEd} \mbox{Nmm}\\
  M_{Rd} &= \g{MRd} > M_{Ed}^{-} = \g{MEdl} \mbox{Nmm}
\end{align*}
\subsection{Beam Local Ductility Check}
\subsubsection{Ratio of compressive reinforcement}
The ratio of compressive reinforcement $\frac{A_{s}'}{A_{s}}$ should be greater
than 0.5. However, this can be skipped since \(A_{s} = A_{s}'\).

\subsubsection{Check $\rho < \rho_{\max}$}
Next is to verify that
\begin{align*}
  \rho &< \rho_{\max} := \rho' +
         \frac{0.0018f_{cd}}{\mu_{\phi}\epsilon_{sy}f_{yd}}\\
  \\[-1cm]               %remove the extra spacing
  \intertext{Where}
  \\[-1cm]               %remove the extra spacing
  \mu_{\phi} &=
               \begin{cases}
                 2q_{0} - 1 & \text{if } T_1 \geq T_{C} \\
                 1 + 2(q_{0} - 1)\frac{T_c}{T_1} & \text{if } T_{1} < T_{C}
               \end{cases}
\end{align*}
However, this check can also be skipped, because $A_s = A_s'$ so $\rho = \rho'$,
and $\rho_{\max}$ is $\rho'$ plus something, namely
$\frac{0.0018f_{cd}}{\mu_{\phi}\epsilon_{sy}f_{yd}}$, let's denote it by $k$.
However, we observe that according to \S EC-8 5.2.2.2, for \emph{multistorey,
  multibay frame building} with DCH, $q_0 = 4.5 \times \frac{\alpha_u}{\alpha_1}
= 4.5 \times 1.3 = 5.85$. This implies that both $2q_0-1$ and
$1+2(q_0-1)\frac{T_c}{T_1}$ are positive, which implies that $\mu_{\phi}$ is
always positive no matter whether $T_1 \geq T_c$ or $T_1 < T_c$, so $k$ is
always positive, so $\rho_{\max} = \rho + k$ is always greater than $\rho$. (i.e
This check can be skipped.)

\subsubsection{Check $\rho > \rho_{\min}$}
Now we calculate $\rho = \frac{A_s}{bd} = \frac{
  \g{As}}{\g{b} \times \g{d}} = \g{rho}$ and $\rho_{\min}$ can
be calculated as:
\begin{align*}
  \rho_{\min} &= 0.5 \frac{f_{ctm}}{f_{yk}} \\
  \\[-1cm]               %remove the extra spacing
  \intertext{Where}
  \\[-1cm]               %remove the extra spacing 
  f_{ctm} &= 0.3 \times f_{ck}^{\frac{2}{3}} \\
  \\[-1cm]               %remove the extra spacing
  \intertext{Therefore}
  \\[-1cm]               %remove the extra spacing
  \rho_{\min} &= 0.5 \frac{0.3 \times 25 ^{\frac{2}{3}}}{500} = 0.0026 < \rho
                \mbox{(verified)}
\end{align*}
\subsection{Shear Strength and Confinement}

\subsubsection{Critical region}
\begin{enumerate}
\item Set $V_{Rd,s} = V_{Ed} = \g{VEd}$ kN.
\item Assuming using a 2-leg hoop with size \g{dh} mm. For the critical
  region, we assume that $\theta = 45 ^{\circ}$ and $\alpha = 90 ^{\circ}$ (i.e.
  the traverse reinforcement (hoops) is vertical).
  \begin{align*}
    s_{required} &= 0.9d \frac{A_{sw}}{V_{Ed}} f_{yd}
                   (\cot \theta + \cot \alpha)\sin \alpha \\
                 &= 0.9d \frac{A_{sw}}{V_{Ed}} f_{yd} (1 + 0) \times 1\\
                 &= 0.9 \times \g{d} \mbox{mm}
                   \times
                   \frac{\g{Asw}\mbox{mm}^{2}}{\g{VEd} \mbox{kN}}
                   \times \g{fyd} \mbox{Nmm}^{-2} \\
                 &= \g{sreq} \mbox{mm}
  \end{align*}
\item According to \S EC-8 5.4.3.1.2 (6)(b) $s$ should not exceed
  \begin{align*}
    s_{\max} &= \min \{ \frac{h_{beam}}{4}, 24\phi_{sw},
               225\mbox{mm}, 8\phi_{long} \} \\
             &= \min \{ \frac{500}{4} \mbox{mm}, 24 \times \g{dh} \mbox{mm},
               225\mbox{mm}, 8 \times 14 \mbox{mm}\} \\
             &= \g{smax}
               \mbox{mm} > s_{required} = \g{sreq} \mbox{mm (Verified)} \\
  \end{align*}
\item To make the spacing practical, we set $s_{actual} = \g{sact}$ mm, and
  the corresponding $V_{Rd, s, actual} $ is
\begin{align*}
  V_{Rd, s, actual}&= 0.9d \frac{A_{sw}}{s_{actual}}f_{yd}\cot  \theta \\
  &= 0.9 \times \g{d} \mbox{mm}
  \times \frac{\g{Asw} \mbox{mm}^{2}}{\g{sact} \mbox{mm}}
  \times \g{fyd} \mbox{Nmm}^{-2} \times 1 \\
  &= \g{VRdsAct} \mbox{kN}
\end{align*}
  which is greater than $V_{Ed} = \g{VEd} $kN (verified).
\end{enumerate}

\subsubsection{Non-Critical Region}
For the non-critical region we assume the following:
\begin{itemize}
\item The struct angle $\theta = 21.8^{\circ}$.
\item The design shear force is same as the critical region: $V_{Ed} =
  \g{VEd}$ kN.
\end{itemize}
It is also assumed that the hoops are the same as the critical region, therefore
we have
\begin{align*}
                 s_{required} &= 0.9d \frac{A_{sw}}{V_{Ed}} f_{yd}
                                (\cot \theta + \cot \alpha)\sin \alpha \\
                              &= 0.9d \frac{A_{sw}}{V_{Ed}} f_{yd} (\cot 21.8^{\circ} +
                                \cot 90 ^{\circ}) \times \sin 90^{\circ}\\
               &= 0.9 \times \g{d} \mbox{mm}
                 \times
                 \frac{\g{Asw}\mbox{mm}^{2}}{\g{VEd} \mbox{kN}}
                 \times \g{fyd} \mbox{Nmm}^{-2}  \times (2.5 + 0) \times 1\\
               &= \g{sreqn} \mbox{mm}
\end{align*}

When considering the non-critical regions, the requirements to be followed are
from EC2 rather than EC8 in terms of spacing and minimum required amount.

According to \S EC2 9.2.2:
\begin{enumerate}
\item $s$ should be smaller than $s_{\max} := \frac{0.75d}{1+\cot\alpha} =
  0.75 \times \g{d} = \g{smaxn} > s_{required}$ (satisfied)
\item the ratio of shear reinforcement ($\rho_w = \frac{A_{sw}}{sb \sin
    \alpha}$) in the non-critical regions should be greater than
\begin{align*}
  \rho_{w,\min} &= \frac{0.08 \sqrt{f_{ck}}}{f_{yk}} = 0.00084 \\
  \rho_w &= \g{rho} > \rho_{w,\min} = 0.00084 \mbox{(satisfied)}
\end{align*}
\item The spacing \g{sreqn} mm is not pratical. For a practical spacing, we
  set $s_{actual} = \g{sactn}$ mm, and the corresponding $V_{Rd, s, actual}$
  is
\begin{align*}
  V_{Rd, s, actual} &= 0.9d \frac{A_{sw}}{s_{actual}}f_{yd}\cot  21.8 ^{\circ} \\
                    &= 0.9 \times \g{d} \mbox{mm} \times \frac{\g{Asw}
                      \mbox{mm}^{2}}{\g{sactn} \mbox{mm}}
                      \times \g{fyd} \mbox{Nmm}^{-2} \times 2.5 \\
                    &= \g{VRdsActn} \mbox{kN}
\end{align*}
  which is greater than $V_{Ed} = \g{VEd}$kN (verified).
\end{enumerate}

\section{Design of Column C1}\label{sec:col}
\subsection{Stage 1: Check Axial Load Ratio}\label{sec:col-axi}
In primary seismic columns the value of the nornlalised axial force $v_{d}$
shall not exceed 0.55 for DCH.Therefore, we have
\begin{align*}
  \nu_{d} &= \frac{N}{bhf_{cd}} \\
          &= \frac{\MyGet{cl.N}}{\MyGet{cl.b}
            \times \MyGet{cl.h}
            \times \MyGet{fcd}} \\
  &= \MyGet{cl.vd} \le 0.55 \mbox{(satisfied)}
\end{align*}

\subsection{Stage 2: Design for Axial-Moment}\label{sec:col-am}
Unlike beams, columns are subjected to both axial and moment actions, so they
must be designed considering moment-axial interation surfaces.

The design of  axial-moment consists of two parts
\begin{enumerate}
\item Select initial longitudinal bars $A_{s,assumed}$
\item For all load combinations, make sure that the triple $[M_{x}, M_{y},N]$
  are within the section's biaxial design envelope.
\end{enumerate}

\renewcommand{\g}[1]{\MyGet{cl.#1}}

\subsubsection{Initial longitudinal bars}
\begin{itemize}
\item The total ratio of longitudinal reinforcement ($\rho_l = \frac{A_s}{bh}$)
  shall be \emph{between 0.01 and 0.04} as per \S EC8 5.4.3.2.2.
\item The spacing between longitudinal bars shall not exceed 200 mm.
\end{itemize}

Assuming $\rho = \frac{A_s}{bh} = \g{rho}$ we have
\begin{align*}
  A_{s,\min} &= \rho b h\\
             &= \g{rho} \times \g{b} \mbox{mm}
               \times \g{h} \mbox{mm} \\
             &= \g{Asmin} \mbox{mm}^2 
\end{align*}
If we use $\phi$ \g{bard} mm bars, we have $A_{s,\mbox{assumed}} = \g{barn} \phi \g{bard}
\mbox{mm} = \g{Asa} \mbox{mm}^2 > A_{s,\min} = \g{Asmin} \mbox{mm}^2$.

\newcommand{\TenBar}{\g{barn} $\phi$ \g{bard} bars}
\subsubsection{Biaxial design envalope check with GaLa}
To verify that the \TenBar{} are Okay. GaLa Reinforcement
$^{\textregistered{}}$ is used as shown in Figure~\ref{fig:gala}. The result
diagram contour is presented in presented in Figure~\ref{fig:envlop}. As shown,
all points are within the envelopes.

\begin{figure}
  \centering
  \includegraphics[width=0.6\textwidth]{gala}
  \caption{The \TenBar{} are modelled in GaLa Reinforcement Version 4.1e}\label{fig:gala}
\end{figure}

\begin{figure}
  \centering
  \MySet{t1.L}{\textcolor{blue}{5}}
\MySet{t1.b}{\textcolor{blue}{0.5}}
\MySet{t1.h}{\textcolor{blue}{1.75}}
\MySet{t1.P}{\textcolor{blue}{100}}
\MySet{t1.k}{\textcolor{blue}{0.833}}
\MySet{t1.I}{\textcolor{blue}{0.2233}}
\MySet{t1.Vy}{\textcolor{blue}{0.0006451}}
\MySet{t1.A}{\textcolor{blue}{0.875}}
\MySet{t1.E}{\textcolor{blue}{3.148 \ensuremath{\times 10^{7}}}}
\MySet{t1.G}{\textcolor{blue}{1.311 \ensuremath{\times 10^{7}}}}
\MySet{t1.Vy.mm}{\textcolor{blue}{0.6451}}
\MySet{t2.E}{\textcolor{blue}{3.148 \ensuremath{\times 10^{0}}}}
\MySet{t2.I}{\textcolor{blue}{0.06337}}
\MySet{t2.L}{\textcolor{blue}{12}}
\MySet{t2.b}{\textcolor{blue}{0.5}}
\MySet{t2.h}{\textcolor{blue}{1.15}}
\MySet{t2.m}{\textcolor{blue}{1380}}
\MySet{t2.fn}{\textcolor{blue}{4.677}}
\MySet{t2.f1}{\textcolor{blue}{0.5602}}
\MySet{t2.f2}{\textcolor{blue}{69.7}}
\MySet{t3.L}{\textcolor{blue}{48}}
\MySet{t3.b}{\textcolor{blue}{1}}
\MySet{t3.h}{\textcolor{blue}{8}}
\MySet{t3.P}{\textcolor{blue}{40}}
\MySet{t3.k}{\textcolor{blue}{0.833}}
\MySet{t3.I}{\textcolor{blue}{42.67}}
\MySet{t3.Vy}{\textcolor{blue}{1.217}}
\MySet{t3.A}{\textcolor{blue}{8}}
\MySet{t3.E}{\textcolor{blue}{2.901 \ensuremath{\times 10^{4}}}}
\MySet{t3.G}{\textcolor{blue}{1.116 \ensuremath{\times 10^{4}}}}
\MySet{t3.Vy.mm}{\textcolor{blue}{1217}}

  \caption{The biaxial design envelope}\label{fig:envlop}
\end{figure}


% \FloatBarrier{}
% \afterpae{\clearpage}
\clearpage
\subsection{Stage 3: Strong-Column-Weak-Beam (SCWB)}
To prevent soft-storey, according to \S EC8 4.4.2.3, for frames $\ge$ 2 storeys,
and allow plastic hinges to form in beams rather than columns, the following
condition should be satisfied at all joints of primary and secondary beams with
primary columns:

\begin{align*}
\sum M_{Rd,c} > 1.3 \sum M_{Rd,b}
\end{align*}

The member data needed are presented in Table~\ref{table:elem}. The value of
$M_{Rd}$ corresponding to the specified axial load is read from
Figure~\ref{fig:envlop}. The axial load (N) that produces the minimum flexural
strength $M_{Rd,c}$ for the column is used. The relavent elements are those
shown in Figure~\ref{fig:zl}. The whole frame is shown in Figure~\ref{fig:za}.

\begin{figure}
  \centering
  \includegraphics[width=0.7\textwidth]{zoommed-all}
  \caption{The whole frame}\label{fig:za}
\end{figure}

\begin{figure}
  \centering
  \includegraphics[width=0.7\textwidth]{zoommed-labeled}
  \caption{The examined elements.}\label{fig:zl}
\end{figure}

\begin{table}[ht]
  \centering
  \begin{tabular}{*{4}{c}}      %ccc
    \toprule
    Element Label & Direction & $N$ (kN) & $M_{Rd}$ (kNm) \\
    \midrule
    B--1--1--c & x & 0         & \g{MRd.bx}  \\ 
    B--1--2--c & x & 0         & \g{MRd.bx}  \\
    C--1       & x & \g{N.c1} & \g{MRd.cx1} \\
    C--2       & x & \g{N.c2} & \g{MRd.cx2} \\
    B--1--1--a & y & 0         & \g{MRd.by}  \\
    B--1--2--a & y & 0         & \g{MRd.by}  \\
    C--1       & y & \g{N.c1} & \g{MRd.cy1} \\
    C--2       & y & \g{N.c2} & \g{MRd.cy2} \\
    \bottomrule
  \end{tabular}
  \caption{Member data for SCWB}\label{table:elem}
\end{table}

Therefore for X direction: $\sum M_{Rd,c} = \g{SMRd.cx} \mbox{kNm} > 1.3 \sum
M_{Rd,b} = \g{SMRd.bx} \mbox{kNm}$ (verified).
For Y direction: $\sum M_{Rd,c} =
\g{SMRd.cy} \mbox{kNm} > 1.3 \sum M_{Rd,b} = \g{SMRd.by} \mbox{kNm}$ (verified).

\renewcommand{\g}[1]{\MyGet{cl.shr.#1}}
\subsection{Stage 4: Shear Design}
\subsubsection{Find $V_{Ed,col}$}
According to EC8 \S 5.4.2.3 (1):
\begin{quote}
  In prilnary seismic columns the design values of shear forces shall be
  determined in accordance with the capacity design rule, on the basis of the
  equilibrium of the column under end moment $M_{i,d}$ (with i= 1,2 denoting the
  end sections of the colmnn), corresponding to plastic hinge formation for
  positive and negative directions of seismic loading.
\end{quote}
The End moment $M_{i,d}$ corresponding plastic hinge formation is

\newcommand{\x}{\min (1,\frac{\sum M_{Rd,col}}{\sum M_{Rd,beam}})}
\begin{equation*}
  M_{id} = \gamma_{Rd} M_{Rd,c,i} \x
\end{equation*}
Where, according to EC8 \S 5.4.2.3 (2)

\begin{itemize}
\item $\gamma_{Rd}$ is the factor accounting for overstrength due to steel
  strain hardening and confinement of the concrete of the compression zone of
  the section, taken as being equal to 1.1.
\item $M_{Rc,i}$ is the design value of the column moment of resistance at end i
  in the sense of the seisnlic bending InOlnent under the considered sense of
  the seismic action;
\item The coefficient $\x$ can be conservatively assumed 1.0.
\end{itemize}

Accordingly, the column shear demand $V_{Ed,col}$ can be computed with respect
to capacity design as follows:
\begin{align*}
  V_{Ed,col} = \gamma_{Rd} \frac{M_{Rd,c,bott} + M_{Rd,c,top}}{L_{clear}}
  &= 1.1 \times \frac{\g{MRdc} \mbox{kNm} +\g{MRdc} \mbox{kNm}}{\g{Lclr} \mbox{mm}} = \g{VEdcol} \mbox{kN}\\
\end{align*}
Where, to be conservative, the load combination with axial load that results in
the largest value of $M_{Rd,c}$ is chosen in order to \emph{increase} the
resulting shear demand $V_{Ed,col}$ and shear reinforcement.

Next we carry out the shear design we have done for beams. Although the
criterion for $s_{\max}$ is specified as per \S EC8 5.4.3.2.2, which is
different from that for beams (which is specified in \S EC8 5.4.3.1.2).

\subsubsection{Critical region}
\begin{enumerate}
\item Set $V_{Rd,s} = V_{Ed} = \g{VEd}$ kN.
\item Assuming using a 2-leg hoop with size \g{dh} mm. For the critical
  region, we assume that $\theta = 45 ^{\circ}$ and $\alpha = 90 ^{\circ}$ (i.e.
  the traverse reinforcement (hoops) is vertical).
  \begin{align*}
    s_{required} &= 0.9d \frac{A_{sw}}{V_{Ed}} f_{yd}
                   (\cot \theta + \cot \alpha)\sin \alpha \\
                 &= 0.9d \frac{A_{sw}}{V_{Ed}} f_{yd} (1 + 0) \times 1\\
                 &= 0.9 \times \g{d} \mbox{mm}
                   \times
                   \frac{\g{Asw}\mbox{mm}^{2}}{\g{VEd} \mbox{kN}}
                   \times \g{fyd} \mbox{Nmm}^{-2} \\
                 &= \g{sreq} \mbox{mm}
  \end{align*}
\item According to \S EC-8 5.4.3.2.2 $s$ should not exceed
  \begin{align*}
    s_{\max} &= \min \{
               \frac{b_0}{2}, 175 \mbox{mm}, 8\phi_{long} \} \\
             &= \min \{ \frac{400}{2} \mbox{mm},
               175\mbox{mm}, 8 \times 20 \mbox{mm}\} \\
             &= \g{smax}
               \mbox{mm} > s_{required} = \g{sreq} \mbox{mm (Verified)} \\
  \end{align*}
\item To make the spacing practical, we set $s_{actual} = \g{sact}$ mm, and
  the corresponding $V_{Rd, s, actual} $ is
\begin{align*}
  V_{Rd, s, actual}&= 0.9d \frac{A_{sw}}{s_{actual}}f_{yd}\cot  \theta \\
  &= 0.9 \times \g{d} \mbox{mm}
  \times \frac{\g{Asw} \mbox{mm}^{2}}{\g{sact} \mbox{mm}}
  \times \g{fyd} \mbox{Nmm}^{-2} \times 1 \\
  &= \g{VRdsAct} \mbox{kN}
\end{align*}
  which is greater than $V_{Ed} = \g{VEd} $kN (verified).
\end{enumerate}

\subsubsection{Non-Critical Region}
For the non-critical region we assume the following:
\begin{itemize}
\item The struct angle $\theta = 21.8^{\circ}$.
\item The design shear force is same as the critical region: $V_{Ed} =
  \g{VEd}$ kN.
\end{itemize}
It is also assumed that the hoops are the same as the critical region, therefore
we have
\begin{align*}
                 s_{required} &= 0.9d \frac{A_{sw}}{V_{Ed}} f_{yd}
                                (\cot \theta + \cot \alpha)\sin \alpha \\
                              &= 0.9d \frac{A_{sw}}{V_{Ed}} f_{yd} (\cot 21.8^{\circ} +
                                \cot 90 ^{\circ}) \times \sin 90^{\circ}\\
               &= 0.9 \times \g{d} \mbox{mm}
                 \times
                 \frac{\g{Asw}\mbox{mm}^{2}}{\g{VEd} \mbox{kN}}
                 \times \g{fyd} \mbox{Nmm}^{-2}  \times (2.5 + 0) \times 1\\
               &= \g{sreqn} \mbox{mm}
\end{align*}

When considering the non-critical regions, the requirements to be followed are
from EC2 rather than EC8 in terms of spacing and minimum required amount.

According to \S EC2 9.2.2:
\begin{enumerate}
\item $s$ should be smaller than $s_{\max} := \frac{0.75d}{1+\cot\alpha} =
  0.75 \times \g{d} = \g{smaxn} > s_{required}$ (satisfied)
\item the ratio of shear reinforcement ($\rho_w = \frac{A_{sw}}{sb \sin
    \alpha}$) in the non-critical regions should be greater than
\begin{align*}
  \rho_{w,\min} &= \frac{0.08 \sqrt{f_{ck}}}{f_{yk}} = 0.00084 \\
  \rho_w &= \g{rho} > \rho_{w,\min} = 0.00084 \mbox{(satisfied)}
\end{align*}
\item The spacing \g{sreqn} mm is not pratical. For a practical spacing, we
  set $s_{actual} = \g{sactn}$ mm, and the corresponding $V_{Rd, s, actual}$
  is
\begin{align*}
  V_{Rd, s, actual} &= 0.9d \frac{A_{sw}}{s_{actual}}f_{yd}\cot  21.8 ^{\circ} \\
                    &= 0.9 \times \g{d} \mbox{mm} \times \frac{\g{Asw}
                      \mbox{mm}^{2}}{\g{sactn} \mbox{mm}}
                      \times \g{fyd} \mbox{Nmm}^{-2} \times 2.5 \\
                    &= \g{VRdsActn} \mbox{kN}
\end{align*}
  which is greater than $V_{Ed} = \g{VEd}$kN (verified).
\end{enumerate}



\renewcommand{\g}[1]{\MyGet{cl.#1}}
\subsection{Stage 5 Local-Ductility Check}
Finally, according to \S EC8 5.4.3.2.2, the \emph{critical-regions} in the
columns should have sufficient curvature ductility ($\mu_{\phi}$). This can be
achieved by satisfying the following expression:
\begin{align*}
  \alpha \omega_{wd} &> 30 \mu_{\phi} \nu_{d} \epsilon_{sy,d} \frac{b}{b_{0}} - 0.035\\
   \\[-1cm]               %remove the extra spacing
  \intertext{Where}
  \\[-1cm]               %remove the extra spacing
  \alpha &= \alpha_n \alpha_{s} \\
  \alpha_n &= 1 - \frac{\sum b_i^2}{6b_0h_0} \\
  \alpha_s &= (1- \frac{s}{2b_0})(1 - \frac{s}{2h_0})\\
  \omega_{wd} &= \frac{\mbox{Volume of confineing hoops and cross ties}}
                {\mbox{volume of the concrete core}} \times \frac{f_{yd}}{f_{cd}}\\
  &= \frac{\sum b_i \times \frac{\phi^2\pi}{4}}{b_0h_0s}
\end{align*}

Where
\begin{itemize}
\item $b_c$ is the smallest dimension of the column cross-section.
\item $h_c$ is the largest dimension of the column cross-section.
\item $b_0$ is the smallest dimension of the confined concrete core measured to
  the centerline of the hoops.
\item $h_0$ is the largest dimension of the confined concrete core measured to
  the centerline of the hoops.
\item $b_i$ is the center-to-center distance between any consecutive bars
  engaged by corner of hoops or cross-ties.
\end{itemize}
When calculating $\omega_{wd}$, the volume of confining hoops are estimated as
the total length of all legs and cross-ties multiplied by the cross-sectional
area of hoops. The volume of confined concrete core is $b_0 \times h_0 \times
s$.

In our case
\begin{align*}
  \alpha_n &= 1 - \frac{2 \times \g{bi} + 6 \times
             \g{bi2}}{6 \times \g{b0} \times \g{h0}}  = \g{aln}\\
  \alpha_s &= (1 - \frac{\g{s}}{2 \times \g{b0}})(1-\frac{\g{s}}{2 \times \g{h0}})
             = \g{als}\\
  \alpha &= \g{aln} \times \g{als} = \g{al}\\
  \omega_{wd} &= \frac{(2 \times \g{bi} + 6 \g{bi2}) \times \pi \times
             \frac{\g{phi}^2}{4}}{\g{b0} \times \g{h0} \times \g{s}} \times
                \frac{\g{fyd}}{\g{fcd}}
  &= \g{omwd}\\
  \alpha \omega_{wd} &= \g{al} \times \g{omwd} = \g{alom}\\
  30 \mu_{\phi}\nu_d \epsilon_{sy,d} \frac{b_c}{b_0} - 0.0035 &= 0.04\\
  \g{alom} &> 0.04 \mbox{ (Verified)}\\
\end{align*}

\section*{Acknowledgments}

I'd like to thank the whole teaching team of CEGE0032 for many helpful material,
particularly Karim, who has made a very helpful slides demonstrating the
methodology.
% \cite{lees2010theoretical}
% \bibliography{sample}

\end{document}