
\newcommand{\frametexthere}[1][]{\node at (9,-1) {#1};}
\newcommand{\ypointhere}[1]{\draw (9,4+8*0.05) node[rectangle,loadpt] (#1)  {};}
\newcommand{\xpointhere}[1]{\draw (9+18*0.05,4) node[rectangle,loadpt] (#1) {};}
\newcommand{\framehere}{
  % The frame
  \draw[line width=3pt,rectangle,rounded corners=1cm] (-3,-2) rectangle (18+3,8+3);
  % Horizontal beams
  \foreach \i in {0,4,8}{
    \draw[beam] (0,\i) -- (18,\i);
  }
  % Vertical beams
  \foreach \i in {0,5,9,13,18}{
    \draw[beam] (\i,0) -- (\i,8);
  }
  % Coloumns
  \foreach \i in {0,5,9,13,18}{
    \foreach \j in {0,4,8}{
      \node at (\i,\j) [col,inner sep=2mm] {};
    }
  }
}

% plot the centre of mass with the given name
\newcommand{\centerofmass}{
  \node[circle,loadpt,label=below:c.o.m] at (9,4)  {};
}

\newcommand{\bigYhere}{
  \draw[dim] (-2,0) -- node[above] {$L_Y=8$m} (-2,8);
}
\newcommand{\smallYhere}{
  % Vertical dimensions
  \foreach \x/\y in {
    0/4,4/8}{
    \draw[dim] (-1,\x) -- node[above] {4} (-1,\y);
  }
}
\newcommand{\bigXhere}{
  \draw[dim] (0,10) -- node[above] {$L_X=18$m} (18,10);
}

\newcommand{\smallXhere}{
  % Horizontal dimensions
  \foreach \x/\y/\l in {
    0/5/5,5/9/4,            %it's <start>/<end>/<label>
    9/13/4,13/18/5}{
    \draw[dim] (\x,9) -- node[above] {\l} (\y,9);
  }
}

\tikzstyle{beam}=[line width=5pt,draw=gray!80]
\tikzstyle{col}=[rectangle,fill=gray!80]
\tikzstyle{dim}=[<->,thick,sloped]
\tikzstyle{loadpt}=[fill=black,inner sep=0.5mm]
\newcommand{\myu}{0.5cm}
% \begin{tikzpicture}
  % \draw[style=help lines,step=5*\myu] (0,0) grid (100,100);
  \begin{scope}[x=\myu,y=\myu]
    % The original frame
    \begin{scope}
      \framehere
      \bigXhere
      \smallXhere
      \bigYhere
      \smallYhere
      \coordinate (myframe) at (-3,4.5);

      \centerofmass
      \draw[thick] (9,4+8*0.05) node[rectangle,loadpt]  {}
      -- +(1,2) node[above,text width=12*\myu] {%
        Center of mass (c.o.m) shifted in $Y$-~direction
        for $5\% \times L_Y = 8 \times 0.05 = 0.4$m
      };

      \draw[thick] (9+18*0.05,4) node[rectangle,loadpt]  {}
      -- +(2,-1) node[below,text width=12*\myu] {%
        c.o.m shifted in $X$-~direction
        for $5\% \times L_X = 18 \times 0.05 = 0.9$m
      };
      \frametexthere[Plan View of Given Frame]
    \end{scope}

    % The array that locates the four primative cases
    \begin{scope}
      \matrix[matrix of nodes, row sep=15*\myu,column sep=25*\myu] (theCase) at
      (45,20) {
        a & b \\
        c & d \\
      };
      % The border arount the matrix
      \draw ([shift={(-17,-3)}] theCase.base) coordinate (rec_dl); \draw
      ([shift={(35,30)}] theCase.base) coordinate (rec_ur); \draw[line width=5]
      (rec_dl) -- (rec_dl |- rec_ur) coordinate (rec_ul) -- (rec_ur) -- (rec_ur
      |- rec_dl) coordinate (rec_rd) -- (rec_dl);

      % The arrow from main frame to array
      \draw[->,very thick] (21,4) to[out=0,in=180] (rec_dl);
    
      \node[right] at ([shift={(-15,28.5)}] theCase.base) {
        \begin{minipage}[l]{1.0\linewidth}
          {\large The 4 \emph{primative load cases}}

          \smallskip
          For each storey, apply the
          seismic force $F_k$ for that storey at the following locations to get
          the \emph{primative load cases.}
        \end{minipage}
      };

      \newcommand{\forceDescription}[3]{%
        $L_{#1}$: Seismic force in $#2$-~direction, applied on c.o.m shifted in
        $#3$-~direction}

    
      \newcommand{\fXhere}[1]{ \draw[{latex}-,very thick] (#1) -- +(3,0)
        node[right] {$F_k$}; } \newcommand{\fYhere}[1]{ \draw[{latex}-,very
        thick] (#1) -- +(0,3) node[above] {$F_k$}; }

      % L1: Seismic force in X-direction, shifted in X direction
      \begin{scope}[shift={(theCase-1-1)}]
        \framehere \bigXhere \frametexthere[\forceDescription{1}{X}{X}]
        \xpointhere{aa} %create a node named ya
        \fXhere{aa}
      \end{scope}
    
      % L2: Seismic force in Y-direction, shifted in X direction
      \begin{scope}[shift={(theCase-1-2)}]
        \framehere \bigXhere \frametexthere[\forceDescription{2}{Y}{X}]
        \xpointhere{ab} %create a node named ya
        \fYhere{ab}
      \end{scope}

      % L3: Seismic force in X-direction, shifted in Y direction
      \begin{scope}[shift={(theCase-2-1)}]
        \framehere \bigYhere \frametexthere[\forceDescription{3}{X}{Y}]
        \ypointhere{ba} %create a node named ya
        \fXhere{ba}
      \end{scope}
    
      % L4: Seismic force in Y-direction, shifted in Y direction
      \begin{scope}[shift={(theCase-2-2)}]
        \framehere \bigYhere \frametexthere[\forceDescription{4}{Y}{Y}]
        \ypointhere{bb} %create a node named ya
        \fYhere{bb}
      \end{scope}

    \end{scope}

    % The combinations
    \node[line width=5,draw,anchor=north west] at (0,45) (load_combs) {%
      \begin{minipage}[l]{1.0\linewidth}
        {\large The 9 \emph{actual load cases}}

        \smallskip
        After the 4 \emph{primative load cases} are input into the model. The 9
        \emph{actual load cases} can be generated as follows:
        \begin{enumerate}
        \item $C_1$ no seismic force involved, only live load and dead load.
        \item Group 1
          \begin{itemize}
          \item $C_2:= C_1 +L_1 + 0.3 L_2$
          \item $C_3:= C_1 +L_1 - 0.3 L_2$
          \item $C_4:= C_1 +0.3 L_1 + L_2$
          \item $C_5:= C_1 +0.3 L_1 - L_2$
          \end{itemize}
        \item Group 2
          \begin{itemize}
          \item $C_6:= C_1 +L_3 + 0.3 L_4$
          \item $C_7:= C_1 +L_3 - 0.3 L_4$
          \item $C_8:= C_1 +0.3 L_3 + L_4$
          \item $C_9:= C_1 +0.3 L_3 - L_4$
          \end{itemize}
        \end{enumerate}
      \end{minipage}
    };

    % the final arrow
    \draw[->,very thick] (rec_ul) to[out=180,in=0] (load_combs.north east);
  \end{scope}
% \end{tikzpicture}