

\begin{tikzpicture}[x=1cm,y=1cm]
  % \draw[style=help lines,step=1cm] (-20cm,-90cm) grid (50cm,15cm);
  \tikzstyle{my_larger}=[text width=1.0cm]

  \begin{scope}[node distance=2cm,every text node part/.style={align=center}]
    \tikzstyle{every node}=[draw,text width=2.5cm]
    \tikzstyle{every child}=[level distance = 3cm]
    \tikzstyle{edge from parent}=[<-,thick,draw]
    \tikzstyle{level 1}=[ellipse, sibling distance=13cm]
    \tikzstyle{level 2}=[rectangle, sibling distance=13cm]
    \tikzstyle{level 3}=[ellipse,sibling distance=6cm]
    \tikzstyle{level 4}=[rectangle]
    \tikzstyle{level 5}=[ellipse]

    % Part 1
    \begin{scope}[text width=2cm]
      \node (spec_a) {Spectual Acceleration $S_d(T_i)$};
      \node (spec_d)[above left of=spec_a,node distance=4cm] {Design Response Spectrum $S_d$};
      \node (T_1)[above right of=spec_a, node distance=4cm] {Fundamental Period of Vibration $T_1$};
      \node (gt_T_1) [above of=T_1,text width=6cm,node distance=5cm,rectangle] {%
        \begin{minipage}[r]{1.0\linewidth}
          {Get $T_1$ \S EC-8 4.3.3.2.2(3)}
          \[T_1 = C_t H^{\frac{3}{4}}\]
          Where \textbf{for moment resistant space concrete frames} shorter than 40m :
          \[C_t = 0.075 \]
          and $H$ is the building height.
        \end{minipage}
      };

      
      \begin{scope}[thick, ->]

        % Define a new coordinates
        \coordinate (n1) at ([yshift=.5cm] spec_a.north);
        \uptodown[spec_d]{spec_a}
        \uptodown[T_1]{spec_a}
        \draw (gt_T_1.south) -- (T_1.north);
      \end{scope}
    \end{scope}
    
    % Part 2   
    \begin{scope}[node distance=4cm]
      \node (m)[right of=spec_a,node distance=23cm] {Seismic Mass $m$};
      \node (gt_m)[above of=m,rectangle,text width=10cm, text ragged] {%
        \begin{minipage}[r]{1.0\linewidth}
          {Calc Seismic mass}
          \\[-.7cm]
          \begin{align*}
            m &= G + [ \psi_{Ei, 1}, \psi_{Ei, 2}, \psi_{Ei, 3}, \psi_{Ei, 4} ]
                \begin{bmatrix}
                  Q_1 \\
                  Q_2 \\
                  Q_3 \\
                  Q_4 
                \end{bmatrix}
            \\
            \\[-1cm]               %remove the extra spacing
            \intertext{Where}
            \\[-1cm]               %remove the extra spacing
            \psi_{Ei,i} &= \mbox{The reduction factor for the $i^{th}$ floor}\\
                          &= \phi_i \times \psi_2\\
            G &= \mbox{The dead load} \\
            Q_i &= \mbox{The live load of the $i^{th}$ floor} \\
            \phi_i &= \mbox{The $\phi$ for the $i^{th}$ floor} \\
            \\[-1cm]               %remove the extra spacing
          \end{align*}
        \end{minipage}
      };

      \node (phi2)[above left of=gt_m,
      node distance=7cm,
      text width=5cm,
      yshift=2cm,
      xshift=-1cm]{%
        \begin{minipage}[r]{1.0\linewidth}
          {\myem Get $\psi_2$}

          \S EC-0 Annex A1.2.2 Table A1.1
          for office $\backslash$ domestic: $\psi_2 = 0.3$ 
        \end{minipage}
      };
      \node (G) [text width = 2cm,left of=phi2, node distance=6cm]{
        The dead load $G$
      };
      \node (Q) [ text width = 2cm,left of=G]{
        The live load $Q$
      };

      \node (psi)[right of=phi2, text width=5cm,node distance=9cm]{%
        \begin{minipage}[r]{1.0\linewidth}
          {\myem Get $\phi$}

          \S EC8 4.2.4(2) Table 4.2
          \begin{equation*}
            \phi  =
            \begin{cases}
              1.0 & \mbox{Roof} \\
              0.8 & \mbox{Similar Storeys} \\
              0.5 & \mbox{Independent storeys}  \\
            \end{cases}
          \end{equation*}
          \\[-1cm] %remove the extra spacing
        \end{minipage}
      };

      \begin{scope}[thick, ->]
        \uptodown[phi2]{gt_m}
        \uptodown[psi]{gt_m}
        \uptodown[G]{gt_m}
        \uptodown[Q]{gt_m}
        \sdown[gt_m]{m}
      \end{scope}
    \end{scope}

    % Part 3
    \begin{scope}
      \node[circle,right of=m,node distance=10cm,my_larger] (la) {$\lambda$};
      \node[rectangle,above of=la,node distance=4cm] (gt_la) {%
        The correction factor: \\

        \(
        \lambda =
        \begin{cases}
          0.85 & \mbox{if $T_1 < 2T_C$ and }    \\
          & \mbox{has more than 2 storeys} \\
          \\[-0.2cm]
          1    & \mbox{Else}                    \\
        \end{cases}
        \)
        
      };
      \draw[->,thick] (gt_la) -- (la);
    \end{scope}

    % Part 4
    \begin{scope}
      \tikzstyle{my_c} = [rectangle, text width=4cm]
      \tikzstyle{my_r} = [text width=2cm]

      \matrix[matrix anchor=gt_F_b.north,
      draw=none,
      below of=m,
      column sep=2cm,
      node distance=3cm,
      row sep=1cm
      ]{
        \node[my_c] (gt_F_b) {%
          Calc seismic force $F_b$ (a.k.a Base shear)
          
          $F_b = S_d(T_1) \times m \times \lambda$
        }; &[2cm]
        \node[my_r] (flr_m) {The mass of each floor $\{m_1 \cdots m_n\}$};&
        \node[my_r] (flr_z) {The level of each floor $\{z_1 \cdots z_n\}$};&
        \node[my_c,text width=8cm] (ck_high) {
          \begin{minipage}[r]{1.0\linewidth}
            Verify the following two the make sure that
            \emph{higher order effects} can be ignored (\S EC-8 4.3.3.2.1):
            \begin{enumerate}
            \item $T_2 < 2$s and $T_1 \le 4T_C$
            \item The building's regular in elevation.(as in \S 4.2.3.3)
            \end{enumerate}
            Where $T_C$ is one of the parameters of the \emph{elastic response
              spectrum }and can be found in \S EC-8 3.2.2.2, and $T_1$ is the
            \emph{fundamental period of vibration.}
          \end{minipage}
        };\\
        \node[circle,my_larger] (F_b) {$F_b$}; \\ %2 more '&' ignored.
      };
      % arrows
      \begin{scope}
        \foreach \x in {m,spec_a,la}{%
          \Uptodown[\x]{gt_F_b}
        }
        \uptodown[gt_F_b]{F_b}
      \end{scope}
    \end{scope}

    % Part 5
    \begin{scope}
      % define the coordinate
      \path (flr_m) -- (flr_z) coordinate[midway] (my_midway);

      \node[rectangle] at ([yshift=-5cm] my_midway) (gt_flr_f){%
        \S 4.3.3.2 Calc force for each storey \\
        The force experience by the $k^{th}$ floor is \\
        $F_k = \frac{F_b(z_km_k)}{\sum_{j}z_jm_j}$
      };
      \node[below of=gt_flr_f] (flr_f) {%
        Force for each floor
        $\{F_1 \cdots F_n\}$
      };

      % arrows
      \foreach \x in {flr_m,flr_z,F_b,ck_high}{%
        \Uptodown[\x]{gt_flr_f}
      }
      \uptodown[gt_flr_f]{flr_f}
    \end{scope}

    
  \end{scope}


  \begin{scope}[node distance=2cm,every text node part/.style={align=center},
    shift={([yshift=5cm,xshift=-2cm] spec_d.north)}]
    % \begin{tikzpicture}
  % \draw[style=help lines] (-10cm,-10cm) grid (20cm,15cm);
    \tikzstyle{every node}=[draw,text width=2.5cm]

    \tikzstyle{every child}=[level distance = 3cm]
    \tikzstyle{edge from parent}=[<-,thick,draw]
    \tikzstyle{level 1}=[ellipse, sibling distance=13cm]
    \tikzstyle{level 2}=[rectangle, sibling distance=13cm]
    \tikzstyle{level 3}=[ellipse,sibling distance=6cm]
    \tikzstyle{level 4}=[rectangle]
    \tikzstyle{level 5}=[ellipse]

    \node (gt_T_d) {Get the Design Response Spectrum} [grow'=up]

    child {node (T_e) {Elastic response Spectrum $T_e$}
%{{{ gt_T_e
      child {node (gt_T_e) {
          \S EC-8 3.2.1 Calc the elastic spectrum
        }
        child {node (a_g) {$a_g$ for ULS and DLS}
          child {node[text width=4cm] (gt_a_g) {%
              Get $a_g$ \\
              From the Italian National Institute of Geophysics and Volcanology
              (INGV)
            }
          }
        }
%{{{ S and spec_param
        child {node (S) {Site factor S}
%{{{ gt_S
          child { node[ text width=5cm] (gt_S) {%
              \S EC-8 3.1.2 Decide the ground type using Table 3.1.\\
              \smallskip
              The parameter needed is the \emph{average shear wave velocity}\\
              $V_{s,30} = \frac{30}{\sum_{i=1}^{N}\frac{h_i}{v_i}}$}
%{{{ v_i_h_i
            child[level distance=5cm] {node[text width = 5cm] (v_i_h_i) }}
          }
%}}}
        }
        child {node[text width=3cm] (spec_param) }}
      }
%}}}
    }
    child {node (q) {behavior factor $q$}
%{{{ q
      child[level distance=5cm] {node[text width=5.5cm,
        % every text node part/.style={align=left}
        ] (gt_q) {
          \begin{minipage}{1.0\linewidth}
            \S EC-8 5.2.2.2 Calc $q$\\
            \smallskip
            For RC structures:
            \[q = q_0 k_w \ge 1.5\] 
            where
            \[ q_0 =
            \begin{cases}
              3\frac{\alpha_u}{\alpha_1} & \mbox{For DCM} \\
              4.5\frac{\alpha_u}{\alpha_1} & \mbox{For DCH} \\
            \end{cases}
            \]
            \\
            and $k_w=1$ for frame structures.
          \end{minipage}
          }
          child[level distance=5cm] {
            node[text width = 6cm] (al_u_1) {
              \S EC-8 5.2.2.2 Get the over strength \\
              $\frac{\alpha_u}{\alpha_1} = 1.3$ \\
              for \emph{multistorey,multibay frame building.}
          }
          }
        }
%}}}
    };
% \end{tikzpicture}
  \end{scope}

  \begin{scope}[xshift=10cm,yshift=-25cm]
    

\tikzstyle{edge from parent}=[->,thick, draw=black]
% \begin{tikzpicture}
\begin{scope}[x=1cm,y=1cm]
  \tikzstyle{every node}=[draw, ellipse,text centered]
  % \draw[style=help lines] (-10,20) grid (10,-30);

  \tikzstyle{edge from parent}=[<-,thick,draw]
  \tikzstyle{every child}=[level distance = 2cm]
  \tikzstyle{level 1}=[ellipse, sibling distance=5cm]

  \node[rectangle,anchor=south] (gt_combs) {%
    \begin{minipage}[l]{1.0\linewidth}
      Get the 8 + 1 combinations of load:

      \begin{itemize}
      \item The combination without seismic loads
      \item 8 combinations which includes
        \begin{itemize}
        \item 4 Combinations if the main direction is in $X$
        \item 4 if in $Y$
        \end{itemize}
      \end{itemize}

      \emph{BUT}, because our building is symetric $X$ and in $Y$~-direction.
      So only 2 combinations is needed for each direction. \emph{So in the end
        we only need 2 + 2 + 1 combinations (i.e. 5).} Which are
      \newcommand{\mf}{\ensuremath{G + \Psi_E Q}}
      \newcommand{\mff}[3][+]{\ensuremath{ W + E_{#2} #1 0.3E_{#3}}}
      \begin{itemize}
      \item Without seismic load : \mf \/ (Let's call it $W$)
      \item Two cases for $X$ :
        \begin{enumerate}
        \item \mff{x}{y}
        \item \mff[-]{x}{y}
        \end{enumerate}
      \item Two cases for $Y$ :
        \begin{enumerate}
        \item \mff{y}{x}
        \item \mff[-]{y}{x}
        \end{enumerate}
      \end{itemize}
    \end{minipage}
  } [grow'=up, growth parent anchor=north]
  child{node (E) {The seismic loads $E$}}
  child{node (Psi_E) {the load factor $\Psi_E$}}
  ;

  \newcommand{\twotwoone}{%
    \ensuremath{%
      \cola[2] + 
      \colb[2] +
      \colc[1]
    }
  } 
  \node[text width=5cm,anchor=north] at ([yshift=-2cm] gt_combs.south) (combs) {%
    \begin{minipage}[l]{1.0\linewidth}
      The \twotwoone combinations \\
      \[
        \{
        \cola[C_{X,l}, C_{X,r}], \colb[C_{Y,l}, C_{Y,r}],\colc[C_{0}]
        \}
      \]

      Where $C_{X,l}$ means the combination that the major seismic load comes
      from the $X$~-direction, and the minir seismic load is at the \emph{left}
      of the major load.
      
    \end{minipage}
  };


  % load cases
  \newcommand{\lda}[1][]{%
    \cola[{L_{#1}}]
  }

  \newcommand{\ldb}[1][]{%
    \colb[{L_{#1}}]
  }


  
  \tikzstyle{level 1}=[sibling distance=8cm,anchor=north]
  \node[rectangle] at ([yshift=-2cm] combs.south) (ecc_do) {Consider eccentricity ?}
  [grow'=down]
  child {
    node[rectangle,text width=8cm] (ecc_yes) {%
      \begin{minipage}[l]{1.0\linewidth}
        \S EC-8 4.3.2: Get 9 cases \\[0.2cm]
        Each of the combinations (excluding $\colc[C_{0}]$ the one which has self-weight only)
        should be applied to 4 different locations around the centre of the mass
        to account for the required. However, again, since our building is
        symetrical in $X$ and $Y$~-directions. It is enough to apply each
        combination to 2 different locations (one shifted in $X$~-direction, and
        one shifted in $Y$.)
        Therefore, the result will be $2 \times (\cola[2] + \colb[2]) + \colc[1] = 9$ cases to run.
      \end{minipage}
    }
    child[level distance=4cm]{
      node[text width=6cm] (the_ecc_yes){%
        \begin{minipage}[l]{1.0\linewidth}
          9 load cases to run
          \[
            \begin{matrix}
              \colc[C_{1}] & & & \\
              \lda[{X,r,X}] &  \lda[{X,r,Y}] &
              \lda[{X,l,X}] &  \lda[{X,l,Y}] \\
              \ldb[{Y,r,X}] &  \ldb[{Y,r,Y}] &
              \ldb[{Y,l,X}] &  \ldb[{Y,l,Y}] \\
            \end{matrix}
          \]
          Where $\lda[{X,r,Y}]$ means the load case based on combination $C_{x,r}$,
          and the point of application is shifted in the $Y$~-direction.
        \end{minipage}
      };
    };
  }
  child {node[rectangle,text width=3cm] (ecc_no) {%
      \begin{minipage}[l]{1.0\linewidth}
        Get 5 cases \\[0.2cm]
        All  $ \twotwoone = 5$ combinations applied to the centre of mass, which
        means 5 cases to run.
      \end{minipage}
    };
  };                          %end of children of ecc_do
  
\end{scope}

% Draw a top-to-down line connecting node #1 and #2, with text #3 is placed at
% the mid point of arrow.
\newcommand{\sdowntxt}[4][]{%
  \draw[very thick,->](#1.south) -- node[auto,draw=none,#4] {#3} (#2.north);
  % The nodes are given somehow inside the to operation! When this is done,
  % the node is placed on the middle of the curve or line created by the to
  % operation. The auto option then causes the node to be moved in such a way
  % that it does not lie on the curve,
}

% arrows

\begin{scope}
  \sdown[gt_combs]{combs}
  \sdown[combs]{ecc_do}
  \sdowntxt[ecc_do]{ecc_yes}{Yes}{ }
  \sdowntxt[ecc_do]{ecc_no}{No}{swap}
  \sdown[ecc_yes]{the_ecc_yes}
\end{scope}

\begin{scope}[shift={([xshift=6cm,yshift=-2cm] the_ecc_yes.east)}]
  
\newcommand{\frametexthere}[1][]{\node at (9,-1) {#1};}
\newcommand{\ypointhere}[1]{\draw (9,4+8*0.05) node[rectangle,loadpt] (#1)  {};}
\newcommand{\xpointhere}[1]{\draw (9+18*0.05,4) node[rectangle,loadpt] (#1) {};}
\newcommand{\framehere}{
  % The frame
  \draw[line width=3pt,rectangle,rounded corners=1cm] (-3,-2) rectangle (18+3,8+3);
  % Horizontal beams
  \foreach \i in {0,4,8}{
    \draw[beam] (0,\i) -- (18,\i);
  }
  % Vertical beams
  \foreach \i in {0,5,9,13,18}{
    \draw[beam] (\i,0) -- (\i,8);
  }
  % Coloumns
  \foreach \i in {0,5,9,13,18}{
    \foreach \j in {0,4,8}{
      \node at (\i,\j) [col,inner sep=2mm] {};
    }
  }
}

% plot the centre of mass with the given name
\newcommand{\centerofmass}{
  \node[circle,loadpt,label=below:c.o.m] at (9,4)  {};
}

\newcommand{\bigYhere}{
  \draw[dim] (-2,0) -- node[above] {$L_Y=8$m} (-2,8);
}
\newcommand{\smallYhere}{
  % Vertical dimensions
  \foreach \x/\y in {
    0/4,4/8}{
    \draw[dim] (-1,\x) -- node[above] {4} (-1,\y);
  }
}
\newcommand{\bigXhere}{
  \draw[dim] (0,10) -- node[above] {$L_X=18$m} (18,10);
}

\newcommand{\smallXhere}{
  % Horizontal dimensions
  \foreach \x/\y/\l in {
    0/5/5,5/9/4,            %it's <start>/<end>/<label>
    9/13/4,13/18/5}{
    \draw[dim] (\x,9) -- node[above] {\l} (\y,9);
  }
}

\tikzstyle{beam}=[line width=5pt,draw=gray!80]
\tikzstyle{col}=[rectangle,fill=gray!80]
\tikzstyle{dim}=[<->,thick,sloped]
\tikzstyle{loadpt}=[fill=black,inner sep=0.5mm]
\newcommand{\myu}{0.5cm}
% \begin{tikzpicture}
  % \draw[style=help lines,step=5*\myu] (0,0) grid (100,100);
  \begin{scope}[x=\myu,y=\myu]
    % The original frame
    \begin{scope}
      \framehere
      \bigXhere
      \smallXhere
      \bigYhere
      \smallYhere
      \coordinate (myframe) at (-3,4.5);

      \centerofmass
      \draw[thick] (9,4+8*0.05) node[rectangle,loadpt]  {}
      -- +(1,2) node[above,text width=12*\myu] {%
        Center of mass (c.o.m) shifted in $Y$-~direction
        for $5\% \times L_Y = 8 \times 0.05 = 0.4$m
      };

      \draw[thick] (9+18*0.05,4) node[rectangle,loadpt]  {}
      -- +(2,-1) node[below,text width=12*\myu] {%
        c.o.m shifted in $X$-~direction
        for $5\% \times L_X = 18 \times 0.05 = 0.9$m
      };
      \frametexthere[Plan View of Given Frame]
    \end{scope}

    % The array that locates the four primative cases
    \begin{scope}
      \matrix[matrix of nodes, row sep=15*\myu,column sep=25*\myu] (theCase) at
      (45,20) {
        a & b \\
        c & d \\
      };
      % The border arount the matrix
      \draw ([shift={(-17,-3)}] theCase.base) coordinate (rec_dl); \draw
      ([shift={(35,30)}] theCase.base) coordinate (rec_ur); \draw[line width=5]
      (rec_dl) -- (rec_dl |- rec_ur) coordinate (rec_ul) -- (rec_ur) -- (rec_ur
      |- rec_dl) coordinate (rec_rd) -- (rec_dl);

      % The arrow from main frame to array
      \draw[->,very thick] (21,4) to[out=0,in=180] (rec_dl);
    
      \node[right] at ([shift={(-15,28.5)}] theCase.base) {
        \begin{minipage}[l]{1.0\linewidth}
          {\large The 4 \emph{primative load cases}}

          \smallskip
          For each storey, apply the
          seismic force $F_k$ for that storey at the following locations to get
          the \emph{primative load cases.}
        \end{minipage}
      };

      \newcommand{\forceDescription}[3]{%
        $L_{#1}$: Seismic force in $#2$-~direction, applied on c.o.m shifted in
        $#3$-~direction}

    
      \newcommand{\fXhere}[1]{ \draw[{latex}-,very thick] (#1) -- +(3,0)
        node[right] {$F_k$}; } \newcommand{\fYhere}[1]{ \draw[{latex}-,very
        thick] (#1) -- +(0,3) node[above] {$F_k$}; }

      % L1: Seismic force in X-direction, shifted in X direction
      \begin{scope}[shift={(theCase-1-1)}]
        \framehere \bigXhere \frametexthere[\forceDescription{1}{X}{X}]
        \xpointhere{aa} %create a node named ya
        \fXhere{aa}
      \end{scope}
    
      % L2: Seismic force in Y-direction, shifted in X direction
      \begin{scope}[shift={(theCase-1-2)}]
        \framehere \bigXhere \frametexthere[\forceDescription{2}{Y}{X}]
        \xpointhere{ab} %create a node named ya
        \fYhere{ab}
      \end{scope}

      % L3: Seismic force in X-direction, shifted in Y direction
      \begin{scope}[shift={(theCase-2-1)}]
        \framehere \bigYhere \frametexthere[\forceDescription{3}{X}{Y}]
        \ypointhere{ba} %create a node named ya
        \fXhere{ba}
      \end{scope}
    
      % L4: Seismic force in Y-direction, shifted in Y direction
      \begin{scope}[shift={(theCase-2-2)}]
        \framehere \bigYhere \frametexthere[\forceDescription{4}{Y}{Y}]
        \ypointhere{bb} %create a node named ya
        \fYhere{bb}
      \end{scope}

    \end{scope}

    % The combinations
    \node[line width=5,draw,anchor=north west] at (0,45) (load_combs) {%
      \begin{minipage}[l]{1.0\linewidth}
        {\large The 9 \emph{actual load cases}}

        \smallskip
        After the 4 \emph{primative load cases} are input into the model. The 9
        \emph{actual load cases} can be generated as follows:
        \begin{enumerate}
        \item $C_1$ no seismic force involved, only live load and dead load.
        \item Group 1
          \begin{itemize}
          \item $C_2:= C_1 +L_1 + 0.3 L_2$
          \item $C_3:= C_1 +L_1 - 0.3 L_2$
          \item $C_4:= C_1 +0.3 L_1 + L_2$
          \item $C_5:= C_1 +0.3 L_1 - L_2$
          \end{itemize}
        \item Group 2
          \begin{itemize}
          \item $C_6:= C_1 +L_3 + 0.3 L_4$
          \item $C_7:= C_1 +L_3 - 0.3 L_4$
          \item $C_8:= C_1 +0.3 L_3 + L_4$
          \item $C_9:= C_1 +0.3 L_3 - L_4$
          \end{itemize}
        \end{enumerate}
      \end{minipage}
    };

    % the final arrow
    \draw[->,very thick] (rec_ul) to[out=180,in=0] (load_combs.north east);
  \end{scope}
% \end{tikzpicture}
\end{scope}




\begin{scope}
  \tikzstyle{every node}=[draw=black, ellipse,text centered,text width=4cm]
  \tikzstyle{every child}=[level distance = 3cm]
  \tikzstyle{level 2}=[rectangle, sibling distance=13cm]

  \node (run) [rectangle,below of=the_ecc_yes, node distance=4cm] {Input these load cases into GSA

    and run.};
  \newcommand{\getfour}[2][]{\{#2_{#1,1},#2_{#1,2},#2_{#1,3},#2_{#1,4}\}}
  \node (gsa_result) [node distance=4cm, below of=run]{
    \begin{minipage}[l]{1.0\linewidth}
      For each load case, we get the displacement of the structural system
      determined by the linear analysis.
      \[
        d_e = \getfour[e]{d}
      \]
    \end{minipage}
  } [grow = down]
  child[level distance=5cm] {
    node[rectangle, text width=8cm] (gt_ds){
      \begin{minipage}[l]{1.0\linewidth}
        \S EC-8 4.3.4 (1) Calc the inelastic displacement
        induced by the design seismic action from the elastic one;
          \begin{align*}
            d_s &= d_e \times q \\
            \\[-1cm]               %remove the extra spacing
            \intertext{Where}
            \\[-1cm]               %remove the extra spacing
            q &: \mbox{the behavior factor from d18}
          \end{align*}
      \end{minipage}
    } child {
      node (ds) {
        \[
          d_s = \getfour[s]{d}
        \]
      } child{ node[rectangle,text width=7cm] (ck_second_ord) {
          \begin{minipage}[l]{1.0\linewidth}
            {Check for 2$^{nd}$ order effects }\\
            \smallskip
            to see whether we can design the structure based upon the load
            cases we have been used so far.
          \end{minipage}
        } edge from parent [->]
        child[level distance =5cm]{ node[rectangle, text width=8cm] (gt_dri) {
            \begin{minipage}[l]{1.0\linewidth}
              \S 4.3.4 Calc the \emph{interstorey drift\/} $d_r$, evaluated as the
              difference of the average lateral displacement $d_s$ at the top
              and bottom of the storey under consideration
              \[
                d_{r,i} = d_{s,top,i} - d_{s,bottom,i}
              \]
              Where $d_{s,top,i}$ is the $d_{s}$ of the storey above the
              $i^{th}$ storey, and $d_{s,bottom,i}$ is the $d_{s}$ of the storey
              below the $i^{th}$ storey. 
            \end{minipage}
          } edge from parent [->];
        };
      }child{ node[rectangle, text width=7.5cm] (ck_SLS){
          \begin{minipage}[l]{1.0\linewidth}
            Check the Damage limit state (DLS):
            Use the reduced displacement $d_{s,reduced} := d_s \nu $
            Where
            \[
              \nu := 
              \begin{cases}
                0.5 & \mbox{For importance class I and II}\\
                0.4 & \mbox{For importance class III and IV}\\
              \end{cases}
              \]
          \end{minipage}
        } edge from parent [->];
      }
    }
  };


  \matrix (the_four) [matrix anchor=dri.north, draw=none,
  column sep=2cm] at ([yshift=-3cm] gt_dri.south) {
    \node (tot_grav) [text width=4cm]{
      The total gravity load \emph{at and above} each storey
      \[ \getfour[tot]{P} \]
    }; &
    \node (dri) {
        The $d_r$ for each storey
        \[
          \getfour[r]{d}
        \]
    }; &
      \node (hi) {
        The interstorey height $\{h_1, h_2, h_3, h_4\}$
      }; &
      \node (tot_ses_shr) {
        The total seismic shear $V_{tot} = F_b$ (See d11)
      };\\
  };

  \uptodown[gt_dri]{dri}
  \node[below of=the_four, node distance=6cm, rectangle] (gt_th){
    \begin{minipage}[l]{1.0\linewidth}
      \S EC-8 4.4.2.2 (2) For each storey, calc the \emph{interstorey drift sensitivity coefficient}
      \[
        \theta_i = \frac{P_{tot,i}
          d_{r,i}}{V_{tot}h_i}
        \]
    \end{minipage}
  } [grow=down] child { node[text width=3cm] (th) {
      $\{ \theta_1, \theta_2, \theta_3, \theta_4\}$
    } child { node[rectangle, text width =10cm] (ck_th){
        \begin{minipage}[l]{1.0\linewidth}
          For each $\theta$ check
          \[
            \begin{cases}
              \mbox{OK} & \mbox{if} \quad \theta < 0.1 \\
              \mbox{A value $\lambda$ should be applied to this storey}
              & \mbox{if}  \quad 0.1 < \theta < 0.2 \\
              \mbox{NOPE, second order effect can't be ignored} & \mbox{otherwise}\\
            \end{cases}
          \]
        \end{minipage}
      } edge from parent [->];
    }
  };

  \foreach \x in {tot_grav, dri, hi, tot_ses_shr}{
    \uptodown[\x]{gt_th}
  }
  \uptodown[run]{gsa_result}
  \uptodown[the_ecc_yes]{run}

\end{scope}


% \end{tikzpicture}
  \end{scope}

  \begin{scope}

  \draw[very thick, ->](the_ecc_yes.east) to[out=0] node[above,sloped,draw=none]{
    How to do it in GSA?} (myframe);

  \draw[very thick, ->](flr_f) to[out=270,in=90] node[above,sloped,draw=none]{
    They are $E$} (E);

  \Uptodown[G]{gt_combs}
  \Uptodown[Q]{gt_combs}
    \Uptodown[gt_T_d]{spec_d}     %connect d and d2

    \newcounter{x}
    % label the procedures
    \foreach \x in {gt_T_1,gt_m,gt_la,gt_F_b, gt_flr_f,
      gt_T_d,gt_T_e,gt_S,gt_q,gt_a_g, run, ck_second_ord, gt_dri, gt_th,
    ck_th, ck_SLS, gt_ds}{ \stepcounter{x}
      \node[fill=gray!20,circle] at (\x.south east) {p\thex}; }
    % label the data
    \setcounter{x}{0}

    \foreach \x in {spec_d,spec_a,T_1,m,phi2,psi,la,flr_m, flr_z,ck_high,F_b,
      a_g,v_i_h_i,spec_param,al_u_1,T_e,S,q, G, Q, gsa_result, dri, tot_grav, hi,
      tot_ses_shr,th, ds}{ \stepcounter{x}
      \node[fill=gray!80,circle,text=white] at ([shift={(0.2cm,-0.1cm)}] \x.south east)
      {d\thex}; }
  \end{scope}

\end{tikzpicture}