
\newcommand\uptodown[2][]{\draw(#1.south)..controls + (0,-.5cm) and + (0,.5cm).. (#2.north);}
\newcommand\righttodown[2][]{\draw(#1.east)..controls + (.5cm,0) and + (0,.5cm).. (#2.north);}
\newcommand\sdown[2][]{\draw(#1.south)--(#2.north);} %Straight down
\newcommand\mycol[1]{\ensuremath{ \{ #1_{1}, \ldots , #1_{n} \} }}           %my collection
\def\myem{\Large \bf}

\begin{tikzpicture}
  % \draw[style=help lines,step=1cm] (-5,-10) grid (40,10);
  % the diagram

  % the border
  \begin{scope}[x=1cm,y=1cm]
    \draw[thick,dashed] (1.5,-9) rectangle (30,9);
  \end{scope}

  % Title
  \node at (2cm,10cm)[right] {%
    \Huge How to get the stiffness response (the $\kappa - M$ and $B-M$ plots) for a
    section};

  \begin{scope}[node distance=2cm]

    \tikzstyle{every node}=[draw, rectangle,text centered]

    \node at (0,7cm) (each_sec)[text width=6cm,left]{%
      For a given section.
      \begin{itemize}
      \item rebar
        \begin{description}
        \item[area] how big is the rebar
        \item[location] where is the rebar
        \item[strength] the strength (from the elastic, perfect-plastic curve) 
        \end{description}
      \item FRP
        \begin{description}
        \item[strength] the strength
        \item[thickness] how thick the FRP is
        \end{description}
      \item Dimension
        \begin{description}
        \item[depth]  $d=450$ mm
        \item[breath] $b=200$ mm
        \end{description}
      \end{itemize}
    };

    \node (0,0) (pick_ep)[text width=4cm,left] {%
      Pick a series of $\epsilon$ from 0 - 0.0035, we get:
      
      \mycol{\epsilon}

      Then, for each $\epsilon_i$, do the following.
    };

    \node (ep) [circle,right of=pick_ep,xshift=2.5cm] {$\epsilon$};

    \node at ([yshift=-2.5cm] ep.west) (find_dc) [%
    right,                                        % anchor=west
    text width=8cm,
    text justified,
    node distance=2.5cm] {%
      Find $d_c$
      
      {\footnotesize
        \begin{enumerate}
        \item Assume a $d_c$;
        \item Calculate the net force $f$ from $d_c$ and $\epsilon$;
        \item If $f$ is (approximately) zero: done; Else: Assume a new $d_c$ and
          go to step 2;
        \end{enumerate}
      }
  };

  \node (dc) [circle, below of=find_dc,node distance=3cm] {$d_c$};
  \node (strn_dist) [ellipse, right of=find_dc, xshift=6cm]{the strain distribution};


%Draw lines

  \begin{scope}[thick, ->]
    \draw (each_sec) -- (pick_ep);
    \draw (pick_ep) -- (ep);
    \draw (ep) -- (ep |- find_dc.north);
    \draw (find_dc.south) -- (dc);
    \draw (dc.east) .. controls +(4cm,0) and +(0,-1cm).. (strn_dist.south west);
    \draw (ep.east) .. controls +(8cm,0) and +(0,1cm).. (strn_dist.north west);
  \end{scope}

  
  \end{scope}

% The section
\begin{scope}[x=.1mm,y=.1mm,shift={([xshift=0.4cm, yshift=-5.5cm]strn_dist.south)}]
  \draw[thick,double] (strn_dist.south) -- ++(0,-600) -- ++(900,0) |- (strn_dist.east);
  % Draw the section
  \begin{scope}
    
    \draw (0,0) rectangle (200,450);
    \draw (50,50) circle (25);
    % Implicitly calculate the point
    \draw (200-50,50) circle (25);
    \draw[fill=gray!40] (0,300) rectangle (200,450);
    \draw[thick, {latex}-] (100,400) .. controls +(50,15) .. 
    (250,470) node[anchor=south] {{\scriptsize The compression zone}};
  \end{scope}

  % The strain diagram
  \begin{scope}[xshift=5cm]
    \draw (0,0) -- ++(0,450) -- ++(100,0) -- (-200,0);
    \draw[thick,-{latex}] (0,0) -- (-200,0) node[pos=0.5,anchor=north]{$\epsilon$};
  \end{scope}

  % The dashed line
  \draw[dashed,thick] (0,300) -- (500,300);
  % The d_c line 
  \draw[{latex'}-{latex'}] (620,300) -- ++(0,150) node[pos=0.5,right,
  text width=2cm,
  text justified]{%
    {\scriptsize Depth of

      Neutral Axis $d_c$}};

  \draw[thick,{latex'}-{latex'}] (500,470) -- ++(100,0) node[pos=0.5,above]{%
    {$\epsilon_{cm}$}};
\end{scope}

% The diagram part 2
\begin{scope}[node distance=2cm]
  \tikzstyle{every node}=[draw, rectangle,text centered]


  % the squares
  \begin{scope}[text width=4cm]
    \node[above of=strn_dist,node distance=3cm] (gt_str_dist){%

      Calculate the stress distribution from the sections properties and
      the strain distribution.

    };
    \node[above of=gt_str_dist,ellipse] (str_dist){%
      the $\sigma$ across the section.
    };
    \node[above right of=str_dist,node distance=4cm] (gt_M) {%
      Find $M$

      From the stress distribution
      Calculate the moment $M$.
    };
    \node[above of=gt_M] (gt_k) {%
       Find $\kappa$
      $$\kappa = \frac{\epsilon_{cm}}{d_c}$$
    };
  \end{scope}

  \node[circle,right of=gt_M,node distance=4cm] (M) {$M$};
  \node[circle,right of=gt_k,node distance=4cm] (k) {$\kappa$};
  \node[below right of=M] (gt_B) {%
    Calculate $ B = \frac{M}{\kappa} $
  };
  \node[circle,right of=gt_B,node distance=3cm] (B){$B$};
  

  % the result node
  \node[above right of=B,node distance=3cm, text width=3cm] (kmB) {%
    For each $\epsilon$ we get three numbers:
    $\{B, M$ and $ \kappa\}$
  };

  \node[right of=kmB,text width = 5cm, node distance=6cm] (kmB_plots){%

    A series of $epsilon$ give us a series of $B,
    M$ and $\kappa$. So we got
    \foreach \x in {M,\kappa,B}{

      \mycol{\x}
    }

    From these, we can plot $M-\kappa$ and $B-M$.
  };

  % arrows
  \begin{scope}[->,thick]

    \draw (strn_dist) -- (gt_str_dist);
    \draw (gt_str_dist) -- (str_dist);
    \foreach \x in {M,k}{
      \draw (str_dist.north) |- (gt_\x.west);
    }

    \foreach \x in {M,k,B}{
      \draw (gt_\x) -- (\x);
      % \draw (\x) .. controls +(1cm,0) and (-1cm,0) .. (kmB);
      \draw (\x) -- (kmB);
    }

    \righttodown[k]{gt_B}
    \righttodown[M]{gt_B}

    \draw (kmB) -- (kmB_plots);
    \draw (each_sec.east) .. controls +(5cm,0) and +(-1cm,0) .. (gt_str_dist.west);
  \end{scope}

\end{scope}


\end{tikzpicture}